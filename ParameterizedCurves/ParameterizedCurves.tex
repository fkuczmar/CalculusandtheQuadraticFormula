\documentclass{ximera}
\title{Parameterized Curves}

\newcommand{\pskip}{\vskip 0.1 in}

\begin{document}
\begin{abstract}
An introduction to describing lines and segments parametrically.
\end{abstract}
\maketitle

You already know two ways to describe curves algebraically - as graphs of functions like $y=x^2$, or as graph of relations like $x^2 + y^2 = 3$. In this section, we introduce a third way and describe curves parametrically. 

To describe a curve in the plane \emph{parametrically} means to express the $x$ and $y$-coordinates as functions of a third variable called the \emph{parameter}. Often, but not always, the parameter measures time, and in this case the parameterization describes a \emph{motion}. It tells us the position of a point as a function of time. As such, the parameterization gives us much more information than just the path of the motion.

For example, it is relatively easy to describe the \emph{path} of the earth around the sun as an ellipse with one focus at the sun, having a certain shape and size. But is would be much more difficult to parameterize the motion of the earth around the sun, as this would require knowing the position of the earth at any time in its orbit.

In this class we will parameterize several different types of motions:

a. motion at a constant velocity

\pskip

b. projectile motion in a uniform gravitational field

\pskip

c. uniform circular motion

\pskip

d. motions around an ellipse

\pskip

This chapter focuses on the first, motion with a constant velocity.

\section{Motion with a Constant Velocity}

In colloquial English, velocity is a synonym for speed, as in ``A 747 has a maximum velocity of 650 miles/hour." But this is incorrect, as velocity is propery a vector that describes both speed and direction. So we might describe the velocity of a car by saying that it is moving due north at a speed of 60 miles/hour. For an object (in the plane) to move with a constant velocity means that it is travelling in a fixed direction at a constant speed.




\begin{exercise}  
  Which of the following define a curve parametrically?  
  \begin{multipleChoice}  
    \choice{$y=x^2$}  
    \choice{$x^2 + y^2 = 5$}  
    \choice[correct]{$x=t^3+t$, $y=2t^4-t$, $0\leq t \leq 1$}  
  \end{multipleChoice}  
\end{exercise}  

\begin{exploration}\label{exp:orth2}
Let $W =\mbox{span}\left({\bf v}_1,{\bf v}_2\right)$. $W$ is a plane through the origin in $\mathbb{R}^3$.  Use the navigation bar at the bottom of the interactive window to go through the steps of constructing an orthogonal basis for $W$.  RIGHT-CLICK and DRAG to rotate the image for a better view.
 
\pdfOnly{
Access GeoGebra interactives through the online version of this text at
 
\href{https://ximera.osu.edu/oerlinalg}{https://ximera.osu.edu/oerlinalg}.
}
 
\begin{onlineOnly}
    \begin{center}
\geogebra{zghsfkym}{900}{600}
\end{center}
\end{onlineOnly}
\end{exploration}



Units should play a more prominent role in our classes, particularly in places where they seem not to play any role at all.

The quadratic formula, for example, comes without units. But in any application the coefficients have meaning and it pays to at least verify the formula is dimensionally correct. And this kind of dimensional thinking can lead to some unexpected insights. %might change the way you look at the quadratic formula.

In its most basic application, the formula
\[
   x = -\frac{b}{2a}\pm \frac{\sqrt{b^2-4ac}}{2a}
\]
tells us how to locate the $x$-intercepts of the parabola 
\begin{equation}
 y=ax^2+bx+c.   \label{Eq:Parabola}
 \end{equation}
 The coordinates $x,y$ and the constant term $c$ have dimensions of length. The coefficient $b$ is dimensionless and gives the slope of the tangent line at the $y$-intercept $Q(0,c)$. Since the product $ax^2$ is a length, so is the reciprocal $a^{-1}$ of the leading coefficient. This reciprocal has several interpretations, each describing the parabola's size. One is related to the parabola's radius of curvature, $(2a)^{-1}$, at the vertex and another to the focal length, $1/(4a)$. But a third, our focus here, is encoded in the quadratic formula itself.


%It determines the parabola's size. Thhe parabola's focal length is $(4a)^{-1}$ and its radius of curvature at the vertex is $(2a)^{-1}$. 

In moving from $Q$ to the axis of symmetry along the normal (with slope $m_\perp = -1/b$) at $Q$, the $x$-coordinate increases by $\Delta x = -b/2a$ and the $y$-coordinate by
\begin{equation}
 \Delta y = m_\perp \Delta x = \left( -\frac{1}{b}\right) \left(- \frac{b}{2a}  \right)= \frac{1}{2a} .  \label{Eq:Subnormal}
\end{equation}
Because $\Delta y$ is independent of $b$, (\ref{Eq:Subnormal}) also records the change in the $y$-coordinate in moving from \emph{any} point $P$ of the parabola (\ref{Eq:Parabola}) to the axis of symmetry along the normal at $P$.

In Leibniz's terminology, the parabola $a^* x=y^2$ has a constant \emph{subnormal} $2a^*$. At a point $P$ on a given curve, the subnormal is the \emph{signed} distance $ON$ from $O$, the foot of the perpendicular from $P$ to the $x$-axis, to $x$-intercept of the normal at $P$. %the point $N$ where the normal at $P$ intersects the $x$-axis.
(Reference: David Dennis, Jere Confrey, Functions of a Curve :
Leibniz's Original Notion of Functions and its Meaning for the Parabola, \emph{CMJ}, March 1995, v.26 (2), pp.124-130.) 

This property of parabolas, that they have constant subnormals, is encoded twice in the quadratic formula. Once in the first term as we have just seen, and again in the second term because the slopes of the tangents to the parabola (\ref{Eq:Parabola}) at its $x$-intercepts are $\pm \sqrt{b^2-4ac}$. 

NOT SURE WHY THIS DESMOS LINK DOES NOT WORK. FOURTH TRY

NOW THE FIFTH TRY

\begin{exploration}\label{exp:orth1}
Cylinder
\begin{center}
\desmos{4rzcqbw4at}{800}{600}
%\desmos{64a5f6cf8761ebbaaa07aea9}{800}{600}
\end{center}
\end{exploration}

%64a5f6cf8761ebbaaa07aea9
%https://teacher.desmos.com/activitybuilder/custom/64359cf956e5d599cc48e5d8

%https://teacher.desmos.com/activitybuilder/custom/64a5f6cf8761ebbaaa07aea9
https://www.desmos.com/calculator/o4zmtwx6hf

\end{document}
