\documentclass{ximera}
\title{The Suspended Slinky}

\newcommand{\pskip}{\vskip 0.1 in}

\begin{document}
\begin{abstract}
A falling slinky.
\end{abstract}
\maketitle

Here's something so counter-intuitive it seems like it can't possibly be true.

\begin{center}  
\youtube{k5s1cMNTmGs}  
\end{center}


\href{https://www.youtube.com/watch?v=k5s1cMNTmGs}{Slinky}



Or scroll down a bit and watch the video here.

\href{dsffdhttps://vanderbei.princeton.edu/WebGL/Slinky.html}{Vanderbei}

\section{The Hanging Slinky}

Setting aside trying to explain why the lower compressed part of the slinky remains suspended for a while, we can nevertheless determine the suspension time. We need only Newton's second law and Hooke's law for an ideal spring.

A slinky is not an ideal spring. In its relaxed state the coils are touching and it cannot be compressed. More importantly, the slinky is \emph{pretensioned}. So it takes some minimum force to strech the slinky and separate the coils. But we can ignore this complication by recognizing that the weight of the compressed coils at the bottom of a suspended slinky is exactly that minimum force. So it does no harm to ignore this part of the slinky and consider only the part above. This part acts like an ideal spring (albeit one that cannot be compressed in its relaxed state) and obeys Hooke's law.

\begin{onlineOnly}
    \begin{center}
\desmos{zqjjgael5j}{450}{600}  %qvk0mzy26u
\end{center}
\end{onlineOnly}

\href{https://www.desmos.com/calculator/zqjjgael5j}{151: Slinky Photo}



Our first step is to describe the slinky hanging under its own weight. This amounts to finding the \emph{stretching function}
\[
   H = f(L) \, , \, 0\leq L \leq L_0 
\]
for a slinky with relaxed length $L_0$ meters. The function takes as an input the distance (say in meters) of a point on the relaxed suspended slinky from the bottom end (in the absense of a gravitational field) and returns as an output the distance (in meters) from the corresponding point on the stretched slinky (suspended in a uniform gravitational field) to the (stretched) slinky's bottom end. 

For this the key is to use Hooke's on the portion of the relaxed spring between heights $L$ and $L+\Delta L$. This segment acts like an idea spring with spring constant
\[
        k_\Delta  =  \left( \frac{L_0}{\Delta L} \right) k,   
\]
where $k$ (in Newtons/meter) is the spring constant for the entire slinky of length $L_0$ meters. Letting $g$ (in $m/\text{sec}^2$) be the uniform gravitational acceleration and $\rho$ (in kg/meter) the linear density of the spring, this segment of the spring gets stretched by the weight $W =g \rho L$ (in Newtons) of the spring below it. So by Hooke's law, the little stretch is
\[
     \Delta H - \Delta L =  \frac{W}{k_\Delta} = \left( \frac{g\rho L}{k L_0} \right) \Delta L .
\]
Summing these little streches gives the height function
\begin{align*}
    H &= f(L)   \\
        &= L +  \frac{g\rho}{k L_0} \int_0^L L^* \, dL^* \\
         & = L + \frac{g\rho}{2kL_0}L^2 \, , \, 0\leq L \leq L_0 .
\end{align*}

Note the entire length of the stretched slinky is
\begin{align*}
   H_0 &= f(L_0)  \\
          &= L_0 + \frac{g\rho L_0}{2k} \\
          &= L_0 +  \frac{W_0}{2k}  .
\end{align*}
A slinky stretched by its own weight $W_0$ is stretched by the same distance as does an ideal massless spring (with the same spring constant) stretched by a weight $W_0$ attached to its end.



Drag the slider $g$ in Line 5 of the worksheet below from $g=0$ to $g=g_0 = 0.5 \text{m/sec}^2$ to turn on the gravitational field and stretch the slinky. 

\begin{onlineOnly}
    \begin{center}
\desmos{vjjibjkdrz}{450}{600}  %qvk0mzy26u
\end{center}
\end{onlineOnly}

\href{https://www.desmos.com/calculator/vjjibjkdrz}{151: Slinky 3}

Here's a crude attempt to model a slinky stretched under its own weight.

\begin{onlineOnly}
    \begin{center}
\desmos{o1ny9elbie}{450}{600}  %qvk0mzy26u
\end{center}
\end{onlineOnly}

\href{https://www.desmos.com/calculator/o1ny9elbie}{151: Slinky 4}


\section{The Falling Slinky}

\begin{onlineOnly}
    \begin{center}
\desmos{fahf3t461i}{450}{600}  %qvk0mzy26u
\end{center}
\end{onlineOnly}

\href{https://www.desmos.com/calculator/fahf3t461i}{151: Slinky 2}
o1ny9elbie


\section{Approximations with Discrete Masses}

With three equal masses $m$ kg each, spring constants all $k$ (SI units), negative direction pointing downward,
\begin{align*}
 \begin{bmatrix}
y_0\\
y_1 \\
y_2
\end{bmatrix} 
&=
-\frac{1}{2}gt^2 
 \begin{bmatrix}
1\\
1 \\
1
\end{bmatrix} 
+
\frac{3mg}{2k} \cos\left( \sqrt{\frac{k}{m}} t  \right)
 \begin{bmatrix}
1\\
0 \\
-1
\end{bmatrix}   \\ \\
& +  
\frac{mg}{6k} \cos\left( \sqrt{\frac{3k}{m}} t  \right)
 \begin{bmatrix}
1\\
-2 \\
1
\end{bmatrix}
-  
\frac{5mg}{3k}
 \begin{bmatrix}
1\\
1 \\
1
\end{bmatrix}.
\end{align*}

\begin{onlineOnly}
    \begin{center}
\desmos{tnhnhzfnbu}{900}{600}
\end{center}
\end{onlineOnly}

\href{https://www.desmos.com/calculator/tnhnhzfnbu}{142: Falling Springs43}



\end{document}


