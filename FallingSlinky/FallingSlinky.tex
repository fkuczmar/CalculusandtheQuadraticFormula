\documentclass{ximera}
\title{The Suspended Slinky}

\newcommand{\pskip}{\vskip 0.1 in}

\begin{document}
\begin{abstract}
A falling slinky.
\end{abstract}
\maketitle

Here's something so counter-intuitive it seems like it can't possibly be true.

\begin{center}  
\youtube{k5s1cMNTmGs}  
\end{center}


\href{https://www.youtube.com/watch?v=k5s1cMNTmGs}{Slinky}



Or scroll down a bit and watch the video here.

\href{dsffdhttps://vanderbei.princeton.edu/WebGL/Slinky.html}{Vanderbei}


\section{Summary of Main Ideas so Far}

\begin{enumerate}
\item Imagine an elastic band, like on you would use at the gym. Attach on end to a wall, and use your right hand to hold the band horizontally stretched from its relaxed length, say by $\Delta L$ meters. Your right hand exerts some force ${\bf F}$ on the band, a force equal to the tension in the band. Now imagine grasping the midpoint of the band with your left hand. Release your right hand from the band and make sure your left hand stays put. Now your left hand exerts the same force ${\bf F}$ on the left half of the band. And this force ${\bf F}$ stretches the left half by only $\Delta L/2$ meters. So if we assume our bands to obey Hooke's law, this tells us that cutting a band in half doubles the spring constant. Similarly, $k_\lambda = k / \lambda$ for $0< \lambda \leq 1$.

\item A band of uniform density hanging in a uniform gravtiational field stretches under its own weight, say again by $\Delta L$ meters. The top half (of the unstretched band) gets stretched by its own weight and also by the lower half of the band, while the bottom half stretches only under its own weight. How do these stretches compare?

Since the bottom half of the spring has twice the spring constant as the whole spring and just half the mass, we should expect the bottom half to stretch 
\[
    \Delta L_2 =  \left(  \frac{1}{2}\right)^2 \Delta L = \Delta L/4 \text{ meters}. 
\]
That leaves a stretch of 
\[
 \Delta L_1 = 3\Delta L / 4 \text{ meters}
\]
in the top half of the unstretched spring. Of this stretch, the same $\Delta L/4$ meters is due to its own weight. And the remaining $\Delta L/2$ meters of the stretch is due to the weight of the bottom half pulling down on the top half.

Now let $k$ be the spring constant of the full band (measured in Newtons/m), $W$ its weight (in kg). Then since the top half of the bank has spring constant $2k$ and weight $W/2$, we know
\[
  \frac{\Delta L}{2} = \frac{W/2}{2k}
\] 
and
\[
   \Delta L =  \frac{W/2}{k} .
\]

In other words, the stretch in an ideal spring with the same spring constant with half the mass of our spring suspended from its end gives the same stretch as the real spring hanging under its own weight.


\begin{onlineOnly}
    \begin{center}
\desmos{7kykohyoic}{450}{600}  
\end{center}
\end{onlineOnly}

\href{https://www.desmos.com/calculator/7kykohyoic}{Hanging Spring 24}


\end{enumerate}

\section{The Hanging Slinky}

Setting aside trying to explain why the lower compressed part of the slinky remains suspended for a while, we can nevertheless determine the suspension time. We need only Newton's second law and Hooke's law for an ideal spring.

A slinky is not an ideal spring. In its relaxed state the coils are touching and it cannot be compressed. More importantly, the slinky is \emph{pretensioned}. So it takes some minimum force to strech the slinky and separate the coils. But we can ignore this complication by recognizing that the weight of the compressed coils at the bottom of a suspended slinky is exactly that minimum force. So it does no harm to ignore this part of the slinky and consider only the part above. This part acts like an ideal spring (albeit one that cannot be compressed in its relaxed state) and obeys Hooke's law.

\begin{onlineOnly}
    \begin{center}
\desmos{zqjjgael5j}{450}{600}  %qvk0mzy26u
\end{center}
\end{onlineOnly}

\href{https://www.desmos.com/calculator/zqjjgael5j}{151: Slinky Photo}



Our first step is to describe the slinky hanging under its own weight. This amounts to finding the \emph{stretching function}
\[
   H = f(L) \, , \, 0\leq L \leq L_0 
\]
for a slinky with relaxed length $L_0$ meters. The function takes as an input the distance (say in meters) of a point on the relaxed slinky from the bottom end in the absense of a gravitational field and returns as an output the distance (in meters) from the corresponding point on the stretched slinky (suspended in a uniform gravitational field) to the (stretched) slinky's bottom end. 

For this the key is to use Hooke's on the portion of the relaxed spring between heights $L$ and $L+\Delta L$. This segment acts like an idea spring with spring constant
\[
        k_\Delta  =  \left( \frac{L_0}{\Delta L} \right) k,   
\]
where $k$ (in Newtons/meter) is the spring constant for the entire slinky of length $L_0$ meters. Letting $g$ (in $m/\text{sec}^2$) be the uniform gravitational acceleration and $\rho$ (in kg/meter) the linear density of the spring, this segment of the spring gets stretched by the weight $W =g \rho L$ (in Newtons) of the spring below it. So by Hooke's law, the little stretch is
\[
     \Delta H - \Delta L =  \frac{W}{k_\Delta} = \left( \frac{g\rho L}{k L_0} \right) \Delta L .
\]
Summing these little streches gives the height function
\begin{align}
    H &= f(L)   \notag  \\
        &= L +  \frac{g\rho}{k L_0} \int_0^L L^* \, dL^*  \notag \\
         & = L + \frac{g\rho}{2kL_0}L^2 \, , \, 0\leq L \leq L_0 .  \label{Eq:Stretch}
\end{align}

Note the full length of the stretched slinky is
\begin{align*}
   H_0 &= f(L_0)  \\
          &= L_0 + \frac{g\rho L_0}{2k} \\
          &= L_0 +  \frac{W_0}{2k}  .
\end{align*}
So a slinky stretched by its own weight $W_0$ is stretched by the same distance as a massless spring (with the same spring constant) stretched by a weight $W_0/2$ attached to its end.

More generally, we can interpret Equation (\ref{Eq:Stretch}) in a similar way. Let
\[
    \lambda = L/L_0
\]
be the dimensionless parameter that measuring the fractional height of a point on the relaxed slinky. Then 
\begin{align*}
       H  & = L + \frac{g\rho}{2kL_0}L^2 \\
           & =  L +   \left(\frac{g\rho L}{2k} \right) \left(\frac{L}{L_0}\right) \\
           & =  L + \lambda  \left(  \frac{W}{2k} \right) ,
\end{align*} 
where $W=g\rho L$ is the weight of the lower part of the relaxed spring of length $L$ meters.  

I'm confused here. Do I want to write this as
\[
           H = L +\frac{W}{2k/\lambda} 
\]
to point out that the lower $L$ meters of the stretched spring acts exactly as we would expect from (\ref{Eq:Stretch}), like a spring with spring constant $k_L = k/\lambda$?

Or should we also write this as
\begin{align*}
        H  & = L + \frac{g\rho}{2kL_0}L^2 \\
            & =  L + \left( \frac{g\rho L_0}{2k} \right) \left( \frac{L^2}{L_0^2} \right) \\
            & = L + \left( \frac{W_0}{2k} \right) \lambda^2 .
\end{align*}

i think both.

In any event, drag the slider $g$ in Line 5 of the worksheet below from $g=0$ to $g=g_0 = 0.5 \text{m/sec}^2$ to turn on the gravitational field and stretch the slinky. 

\begin{onlineOnly}
    \begin{center}
\desmos{7kykohyoic}{450}{600}  
\end{center}
\end{onlineOnly}

\href{https://www.desmos.com/calculator/7kykohyoic}{Hanging Spring 24}



\begin{onlineOnly}
    \begin{center}
\desmos{vjjibjkdrz}{450}{600}  %qvk0mzy26u
\end{center}
\end{onlineOnly}

\href{https://www.desmos.com/calculator/vjjibjkdrz}{151: Slinky 3}

Here's a crude attempt to model a slinky stretched under its own weight.

\begin{onlineOnly}
    \begin{center}
\desmos{o1ny9elbie}{450}{600}  %qvk0mzy26u
\end{center}
\end{onlineOnly}

\href{https://www.desmos.com/calculator/o1ny9elbie}{151: Slinky 4}


\section{The Falling Slinky, Instantaneous Collapse Model}




\begin{onlineOnly}
    \begin{center}
\desmos{fahf3t461i}{450}{600}  %qvk0mzy26u
\end{center}
\end{onlineOnly}

\href{https://www.desmos.com/calculator/fahf3t461i}{151: Slinky 2}
o1ny9elbie


\section{Approximations with Discrete Masses}

With three equal masses $m$ kg each, spring constants all $k$ (SI units), negative direction pointing downward,
\begin{align*}
 \begin{bmatrix}
y_0\\
y_1 \\
y_2
\end{bmatrix} 
&=
-\frac{1}{2}gt^2 
 \begin{bmatrix}
1\\
1 \\
1
\end{bmatrix} 
+
\frac{3mg}{2k} \cos\left( \sqrt{\frac{k}{m}} t  \right)
 \begin{bmatrix}
1\\
0 \\
-1
\end{bmatrix}   \\ \\
& +  
\frac{mg}{6k} \cos\left( \sqrt{\frac{3k}{m}} t  \right)
 \begin{bmatrix}
1\\
-2 \\
1
\end{bmatrix}
-  
\frac{5mg}{3k}
 \begin{bmatrix}
1\\
1 \\
1
\end{bmatrix}.
\end{align*}

\begin{onlineOnly}
    \begin{center}
\desmos{tnhnhzfnbu}{900}{600}
\end{center}
\end{onlineOnly}

\href{https://www.desmos.com/calculator/tnhnhzfnbu}{142: Falling Springs43}



\end{document}


