\documentclass{ximera}
\title{Disco Dancing, Poncelet's Porism, and Simple Pendulums}

\newcommand{\pskip}{\vskip 0.1 in}

\begin{document}
\begin{abstract}
A twist on the lighthouse problem (related rates) leads to Poncelet's porism and some insights into simple pendulums.
\end{abstract}
\maketitle

\section*{Main Results}

\begin{itemize}

\item{The}

\end{itemize}


\section{Introduction}

\begin{exploration}


\begin{onlineOnly}
    \begin{center}
\desmos{wmyk3ioe3t}{450}{600}  
\end{center}
\end{onlineOnly}

\href{https://www.desmos.com/calculator/wmyk3ioe3t}{Oscillating Pendulum}



\end{exploration}

\begin{exploration}
The chord below bisects the motion of a revolving pendulum. Draw other bisecting chords.

\begin{onlineOnly}
    \begin{center}
\desmos{zlyigzhmpn}{450}{600}  
\end{center}
\end{onlineOnly}

\href{https://www.desmos.com/calculator/zlyigzhmpn}{Revolving Pendulum Opening Problem 1}

\end{exploration}

\begin{exploration}
The chord below bisects the motion of a revolving pendulum. Draw other bisecting chords.

\begin{onlineOnly}
    \begin{center}
\desmos{hqajtdhr31}{450}{600}  
\end{center}
\end{onlineOnly}

\href{https://www.desmos.com/calculator/hqajtdhr31}{Revolving Pendulum Opening Problem 2}

\end{exploration}



\section*{DiscoDancing}

\section*{Poncelet's Porism}


\begin{onlineOnly}
    \begin{center}
\desmos{q16ify3gtm}{450}{600}  
\end{center}
\end{onlineOnly}

\href{https://www.desmos.com/calculator/q16ify3gtm}{Poncelt Metric with Jacobi Functions}



\begin{onlineOnly}
    \begin{center}
\desmos{hm5zop04ob}{450}{600}  
\end{center}
\end{onlineOnly}

\href{https://www.desmos.com/calculator/hm5zop04ob}{Poncelt Metric with Jacobi Functions 2}



\begin{onlineOnly}
    \begin{center}
\desmos{lwbypn9rje}{450}{600}  
\end{center}
\end{onlineOnly}

\href{https://www.desmos.com/calculator/lwbypn9rje}{Poncelets Porism and Jacobi}



\begin{exploration}

Given $\lambda$, $0 < \lambda < 1/2$, find the center and radius of the envelope of the family of $\lambda$-chords. 

\pskip

Let $(0,a)$ be the circle's center and $r$ its radius.

Let $\phi$ be the angle the first iscochronous chord (with one endpoint $(0,-R)$) makes with the vertical. Then
\[
      \phi = \pi/2 - \text{am}(\lambda T) ,
\]
where
\[
  T = \int_0^{\pi} \frac{du}{\sqrt{1-k^2\sin^2 u}}
\] 
is the period of revolution.

Then because the first chord is tangent to the envelope,
\[
     r = (R+a) \sin \phi
\]
And because the envelope belongs to the hyperbolic family generated by the motion,
\[
  \frac{4Ra}{(R+a)^2 - r^2} = k^2 .
\]

Substituting and some algebra leads to
\[
  a =   R \sec \phi \left( \frac{2}{k^2} \sec \phi - \cos\phi \pm   \frac{2}{u} \sqrt{\frac{\sec^2\phi}{k^2} - 1}   \right) .
\]

We choose the minus sign because the plus sign gives the center of the circle in the upper half of the hyperbolic family (the circle that lies outside the pendulum circle.)



\begin{onlineOnly}
    \begin{center}
\desmos{svb8pwijet}{450}{600}  
\end{center}
\end{onlineOnly}

\href{https://www.desmos.com/calculator/svb8pwijet}{Jacobi Elliptic Functions and Theta Functions v 2}


\end{exploration}


\section{Jacobi Closure Formulas}

The polygonal path closes when the time subtended by the chords is a rational multiple $\lambda$ of the period. The gives the euqation
\[
     \int_0^{\arcsin(\frac{r}{R+a}} \frac{du}{(R+a)^2 - r^2 - 4a\sin^2(u/2)} = \lambda \int_0^{2\pi} \frac{du}{(R+a)^2 - r^2 - 4a\sin^2(u/2)} .
\]
Taking $R=1$ gives a relation between $a$ and $r$. We can use desmos to graph these curves as below. But it's hard to see how to use this to get a relation, even when $\lambda=1/3$ (triangle) or $\lambda  = 1/4$ (quadrilateral).


\begin{onlineOnly}
    \begin{center}
\desmos{wnef0tpllt}{450}{600}  
\end{center}
\end{onlineOnly}

\href{https://www.desmos.com/calculator/wnef0tpllt}{Jacobi Formula}



\section*{Revolving Pendulums}


\end{document}