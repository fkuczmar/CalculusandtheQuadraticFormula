\documentclass{ximera}
\title{Poncelet's Porism and Simple Pendulums}

\newcommand{\pskip}{\vskip 0.1 in}

\begin{document}
\begin{abstract}
We investigate the families of \emph{isochronous chords} of a pendulum motion. The chords of each family subtend arcs of equal time.
\end{abstract}
\maketitle

Think of a simple pendulum, and you probably picture an oscillating mass at the end of a string. But this paper is mostly about revolving pendulums. Imagine instead a mass attached to a weightless rod or a small bead that slides without friction around a circluar wire. Imagine also, the mass (or bead) to be launched from its low point with sufficient speed to repeatedly revolve around its circular path. 

Now construct the chords of the circular path that bisect the period in the sense that the pendulum traverses the two arcs subtended by the chord in equal times. One such chord is the vertical diameter. Another is horizontal and lies somewhere above the center. But where? And what can we say about the other bisecting chords.

Or pick some fraction of the period of revolution. How can we draw a chord that subtends an arc traversed in this fraction of the period? What can we say about all such chords and the curve they envelope?

These questions are closely related to classical theorem from projective geometry, or rather one of the proofs of that theorem.



\begin{exploration}
A simple pendulum oscillates as shown below. How can we construct other chords of the arc $AB$ that subtend the same time as chord $\overline{XY}$. What can we say about the envelope of all such chords?

\begin{onlineOnly}
    \begin{center}
\desmos{pleca0vkjw}{450}{600}  
\end{center}
\end{onlineOnly}

\href{https://www.desmos.com/calculator/pleca0vkjw}{Oscillating Pendulum}

\end{exploration}

\begin{exploration}
The horizontal chord below bisects the period of a revolving pendulum. How can we draw other bisecting chords? What can we say about the family of all such chords?

\begin{onlineOnly}
    \begin{center}
\desmos{x5fzsfvxdb}{450}{600}  
\end{center}
\end{onlineOnly}

\href{https://www.desmos.com/calculator/x5fzsfvxdb}{Revolving Pendulum Opening Problem 1}

\end{exploration}

\begin{exploration}
What can we say about the isochronous family generated by the horizontal chord $\overline{XY}$ for the revolving pendulum below? The horizontal chord through $F$ bisects the motion.

\begin{onlineOnly}
    \begin{center}
\desmos{1kyfdn6yyc}{450}{600}  
\end{center}
\end{onlineOnly}

\href{https://www.desmos.com/calculator/1kyfdn6yyc}{Revolving Pendulum Opening Problem 2}

\end{exploration}

All these questions are related to Poncelet's Porism. 

\section*{Poncelet's Porism}

%A spotlight in a circular dance hall rotates at a constant rate as the light beam sweeps around the wall. Where is the light beam when its speed is increasing at the fastest rate?

%Let $R$ be the radius of the dance hall and $a$ the distance between the spotlight and the room's center. Let $\omega$ be the rotation rate (in rad/sec).

%Draw a smaller circle (${\cal C}_2$) inside a larger one (${\cal C}_1$). Then pick a point $P_0$ on ${\cal C}_1$ and construct a polygonal path $P_0 P_1 P_2 \ldots$ of chords $\overline{P_0P_1}$, $\overline{P_0P_1}, \ldots$ tangent to ${\cal C}_2$ running counterclockwise around ${\cal C}_1$. Most likely the path never closes.  But if by chance it does, then it was not because you made a fortunate choice of $P_0$ but because you were lucky in choosing the initial confuguation of circles. The  path then would close for any starting point and do so in the same way. The figure below shows a configuration of ${\cal C}_1$ and ${\cal C}_2$ for which the path closes with five segments and forms a pentagram circumscribed about ${\cal C}_2$.

Draw a smaller circle (${\cal C}_2$) inside a larger one (${\cal C}_1$). Then pick a point $P_0$ on ${\cal C}_1$ and construct a polygonal path $P_0 P_1 P_2 \ldots$ of chords $\overline{P_0P_1}$, $\overline{P_0P_1}, \ldots$ tangent to ${\cal C}_2$ running counterclockwise around ${\cal C}_1$. Most likely the path never closes.  But if by chance it does, then it was not due to your choice of $P_0$ but because you were lucky in choosing the initial confuguation of circles. The path would then close for any starting point and do so in the same way. The figure below shows a configuration of ${\cal C}_1$ and ${\cal C}_2$ for which the path closes with five segments and forms a pentagram circumscribed about ${\cal C}_2$.


This fact is the circular version of Poncelet's Porism (a similar statement holds for two nested ellipses). The only proof I've seen relies on defining a new metric around ${\cal C}_1$, one with the property that all chords tangent to ${\cal C}_2$ subtend equal arclengths of ${\cal C}_1$. 

Defining this metric is like wrapping a number line around ${\cal C}_1$ as illustrated in the figure below. The tick marks are spaced at equal intervals, say one unit apart and here each chord subtends a shorter arc of $8$ units. Since ${\cal C}_1$ has circumference $24$ units, every path, not just the one shown, forms a triangle circumscribed about ${\cal C}_2$.  

\begin{exploration}
Move the slider $\theta_1$ to change the path and slider $a$ to change the distance between the centers of ${\cal C}_1$ and ${\cal C}_2$.
 
\begin{onlineOnly}
    \begin{center}
\desmos{wzctareiln}{450}{600}  
\end{center}
\end{onlineOnly}

\href{https://www.desmos.com/calculator/wzctareiln}{Poncelt Metric with Jacobi Functions}

\end{exploration}

In the example below, each chord subtends $2/5$ of the circumference in the new metric. So every path closes after five segments and forms a pentagram circumscribed about ${\cal C}_2$. More generally, when the chords subtend an arclength equal to a rational multiple $\lambda = q/p$ ($p$ and $q$ relatively prime positive integers) of the circumference, the path closes with $p$ chords and forms a $\{p/q \}$ star polygon. Otherwise, when the ratio $\lambda = \text{arc subtended by chords : circumference}$ is irrational the path never closes.

\begin{exploration}
\begin{onlineOnly}
    \begin{center}
\desmos{qco4rghhfi }{450}{600}  
\end{center}
\end{onlineOnly}

\href{https://www.desmos.com/calculator/qco4rghhfi}{Pentagram}

\end{exploration}

We can better relate the proof of Poncelet's Porism to pendulums by replacing the distance metric around ${\cal C}_1$ with a time metric. Imagine a motion around ${\cal C}_1$ that traverses the shorter arcs subtended by the tangent chords in equal times. A priori, it's not clear such an \emph{equal-time} motion exits, but if so it would prove Poncelet's Porism. Every path, for example, would close with three segments for a motion that takes $1/3$ the period to traverse each arc. %If for example, the motion takes $1/3$ the period to tranverse each arc, then each path would close with three segments.

It's easy to describe such a motion when ${\cal C}_1$ is a point $A$. Then Poncelet's Porism is trivially true (each path closes with two segments) and a motion that rotates about $A$ at a constant rate traverses all the arcs subtended by the chords through $A$ in half the period. But there are many other equal-time motions when ${\cal C}_1$ is a point, some of which are easier to describe algebraically. 

Because the equal-time condition relates only the times to traverse arcs subtended by chords through $A$, we have the freedom to arbitrarily define half the motion on either of the two arcs subtended by a particular chord through $A$. This choice, assumed to be continuous, then determines the motion along the other arc. This motion is traced by the other end $P^\prime$ of the chord $\overline{PP^\prime}$ through $A$ as $P$ follows its arbitrarily defined motion along the chosen arc. This is simply because from any initial position of the chord, the points $P$ and $P^\prime$ arrive simultaneously at the other's starting point.



%Suppose for example, we choose a continous motion 
%\[
%   v_P = f(\phi) \, , 0\leq \phi \leq \pi , 
%\]
%along the semicircle to the right of the vertical diameter $ED$. To extend this to an equal-time motion around the circle, imagine $P$ moving along the semicircle with this motion and let $\overline{PP^\prime}$ be the chord through $A$. Then $P^\prime$ moves with the other half of the equal-time motion. This is simply because from any initial position of the chord, the points $P$ and $P^\prime$ arrive simultaneously at the other's starting point.

To complete the description of the equal-time motion, we need to relate the speeds of $P$ and $P^\prime$. Let $ds$ and $ds^\prime$ be the respective arc lengths traced by $P$ and $P^\prime$ as $\overline{PP^\prime}$ turns through the angle $d\phi$. Then 
\begin{equation}
   ds =  \sqrt{(\rho \, d\phi)^2 + d\rho^2}   =   (\rho \, d\phi) \csc \alpha ,   \label{Eq:ArcLength}
\end{equation}
where $\rho = AP$ is the distance from $A$ to $P$ and $\alpha$ the acute angle $\overline{AP}$ makes with the circle at $P$ . Similarly, because $\overline{AP^\prime}$ cuts ${\cal C}_1$ at the same angle $\alpha$,
\[
   ds =   (\rho^\prime \, d\phi) \csc \alpha ,
\]
where $\rho^\prime = AP^\prime$. And because $P$ and $P^\prime$ traverse these arclengths in equal times,
\[
   \frac{ds}{v_P} = \frac{ds^\prime}{v_{P^\prime}}
\] 
and
\[
    \frac{\rho}{v_P} = \frac{\rho^\prime}{v_{P^\prime}} .
\]
So if 
\[
  v_P = f(\phi) \, , 0\leq \phi \leq \pi , 
\]
is our choice of a continuous motion (ie. a speed function) around one of the two arcs subtended by a chord $\overline{CD}$ through $A$, the function
\[
   v_P = v(\phi)  = 
\begin{cases}
          f(\phi) \, , 0\leq \phi \leq \pi \\
         \frac{\rho (\phi + \pi)}{\rho (\phi)} f(\phi-\pi) \, , \pi < \phi < 2\pi ,
\end{cases}
\]
has the equal-time property. Here $\phi = \angle CAP$.

To make this motion continuous at $\phi=0$ and $\phi= \pi$, we should be sure that $A$ arrives at $D$ (when $\phi=pi$) and the same speed at which $P^\prime$ left $D$ (when $\phi = 0$). That is, we choose $f$ so that 
\[
   f(\pi) =  \frac{\rho(\pi)}{\rho(0)} f(0) .
\]

%Equivalently, any continuous equal-time motion is of the form

We can also express any continuous equal-time motion in the form
\[
      v_P = v(\phi) = \omega_0 \rho (\phi) h(\phi) ,
\]
where $\omega_0$ is a constant with units rad/sec and $h$ is a function with period $\pi$ and dimensionless output. 

For example, dividing both sides of ({\ref{Eq:Arclength}) by the differential time (in seconds) it takes $\overline{PP^\prime}$ to turn through the angle $d\phi$ gives 
\[
    v_P = \frac{ds}{dt} = (\rho \csc \alpha) \frac{d\phi}{dt}.
\]
So taking $h(\phi) = \csc \alpha$ gives the motion that rotates about $A$ at the constant rate of $\omega_0$ rad/sec.


A simpler choice would be to take $h(\phi)=1$ to get the constant-time motion
\[
   v_P =   v(\phi) = \omega_0 \rho(\phi) 
\]
with speed proportional to the distance $\rho(\phi) = AP$. Here $\omega_0$ is the rotation rate of chord $PP^\prime$ when $PP^\prime$ is the vertical diameter through $A$. For this motion, the rotation rate $\omega = \omega(\phi)$ is






To find an expression for the speed of such a motion, consider more generally a motion traced by a point $P$ that rotates about $A$ with rate $\omega = f(\theta)$ rad/sec a function of position (central angle $\angle EOP$). Then as $\overline{AP}$ turns through the differential angle $d\phi$, $P$ traverses the differential arclength
\[
   ds =  \sqrt{(\rho \, d\phi)^2 + d\rho^2}   =   (\rho \, d\phi) \csc \alpha ,
\]
the second equality because $\alpha$ is then angle opposite the leg with length $\rho \, d\phi$ in the right triangle with other leg $d\rho$ and hypotenuse $ds$. Dividing both sides by the differential time $dt$ gives the speed of $P$,
\[
     v_P  = \omega \rho \csc \alpha .
\]





%Then the speed at a point $P$ of the path is  
%\[
%      v_P = \omega \rho \csc \alpha,
%\]
%where $\rho = AP$ is the distance from $A$ to $P$ and $\alpha$ is the acute angle $\overline{AP}$ makes with the circle at $P$. This follows from the expression
%\[
%   ds =  \sqrt{(\rho \, d\phi)^2 + d\rho^2}   =   (\rho \, d\phi) \csc \alpha
%\]
%for the differential arclength in polar coordinates. As $\overline{AP}$ turns through the differential angle $d\phi$, the product $\rho \,d\phi$ measures the component of $P$'s displacement perpendicular to the polar radius $\overline{AP}$ and $d\rho$ the component parallel to $\overline{AP}$. The angle $\alpha$ is opposite the leg with $\rho \, d\phi$ in the right triangle with other leg $s\rho$ and hypotenuse $ds$.



%To describe an equal-time motion when ${\cal C}_1$ is a point $A_0$ (and Poncelet's porism is trivially true) is easy. Just let the motion rotate about $A_0$ at a constant rate.

\begin{onlineOnly}
    \begin{center}
\desmos{owbuu0q8lz }{450}{600}  
\end{center}
\end{onlineOnly}

\href{https://www.desmos.com/calculator/owbuu0q8lz}{Poncelelts Porism Disco Dancing}

%To find an expression for the speed of a general motion, let $\omega$ (in rad/sec) be the rate of rotation (not necessarily contant. The the speed at a point $P$ of the path is
%\[
%      v_P = \omega \rho \csc \alpha,
%\]
%where $\rho = AP$ is the distance from $A$ to $P$ and $\alpha$ is the angle $\overline{AP}$ makes with the circle at $P$. This follows from the expression
%\[
%   ds =  \sqrt{(\rho \, d\phi)^2 + d\rho^2}   =   (\rho \, d\phi) \csc \alpha
%\]
%for the differential arclength in polar coordinates. As $\overline{AP}$ turns through the differential angle $d\phi$, the product $rho , d\phi$ measures the component of $P$'s displacement perpendicular to the polar radius $\overline{AP}$ and $d\rho$ the component parallel to $\overline{AP}$. The angle $\alpha$ is opposite the leg with $\rho \, d\phi$ in the right triangle with other leg $\rho \, d\phi$ and hypotenuse $ds$.


%traversed by $P$ as $\overline{AP}$ turns through the angle $d\phi$ (since the angle $\alpha$ is opposite the leg with $\rho \, d\phi$ in the right triangle with other leg $\rho \, d\phi$ and hypotenuse $ds$).


%To see why, note that as $\overline{AP}$ turns through the differential angle $d\phi$, $P$ traverses the differential arclength
%\[ 
%      ds = (\rho \, d\phi) \csc \alpha .
%\]
%This is because $\rho , d\phi$ measures the component of $P$'s displacement perpendicular to $\overline{AP}$ and in the differential right triangle with legs $d\rho$, $\rho \, d\phi$, and hypotenuse $ds$, the angle $\alpha$ is opposite the leg with length $\rho \, d\phi$.




%To express the speed $v$ of the motion in terms of the counterclockwise angle $\phi = \angle EOP$ from the vertical to segment $OP$ below


%Instead of defining a new metric around ${\cal C}_1$, imagine instead a point moving around ${\cal C}_1$ in such a way that it traverses the shorter arcs subtended by the tangent chords in equal times. 

With $\theta$ the measure of the central angle $\angle EOP$ that $\overline{OP}$ makes with the downward vertical, we have
\[
  \rho = \sqrt{R^2 + a^2 + 2ar \cos \theta}
\]
and 
\[
    v_P = \omega \sqrt{R^2 + a^2 + 2ar \cos \theta} \csc \alpha .
\]

To express $\csc\alpha$ in terms of $\theta$, use the law of sines in $\Delta AOP$ where $\angle APO = \pi/2 - alpha$. This gives
\[
    \frac{\sin\theta}{\rho} = \frac{\cos\alpha}{a}.
\]



\section*{Poncelet's Porism}


\begin{exploration}

\begin{onlineOnly}
    \begin{center}
\desmos{0aihemcv0o}{450}{600}  
\end{center}
\end{onlineOnly}

\href{https://www.desmos.com/calculator/0aihemcv0o}{Poncelts Porism Triangles}

\end{exploration}


\begin{exploration}

\begin{onlineOnly}
    \begin{center}
\desmos{pua4ogjcmr}{450}{600}  
\end{center}
\end{onlineOnly}

\href{https://www.desmos.com/calculator/pua4ogjcmr}{Poncelts Porism Quadrilaterals}


\end{exploration}



\begin{onlineOnly}
    \begin{center}
\desmos{q16ify3gtm}{450}{600}  
\end{center}
\end{onlineOnly}

\href{https://www.desmos.com/calculator/q16ify3gtm}{Poncelt Metric with Jacobi Functions}



\begin{onlineOnly}
    \begin{center}
\desmos{hm5zop04ob}{450}{600}  
\end{center}
\end{onlineOnly}

\href{https://www.desmos.com/calculator/hm5zop04ob}{Poncelt Metric with Jacobi Functions 2}



\begin{onlineOnly}
    \begin{center}
\desmos{lwbypn9rje}{450}{600}  
\end{center}
\end{onlineOnly}

\href{https://www.desmos.com/calculator/lwbypn9rje}{Poncelets Porism and Jacobi}



\begin{exploration}

Given $\lambda$, $0 < \lambda < 1/2$, find the center and radius of the envelope of the family of $\lambda$-chords. 

\pskip

Let $(0,a)$ be the circle's center and $r$ its radius.

Let $\phi$ be the angle the first iscochronous chord (with one endpoint $(0,-R)$) makes with the vertical. Then
\[
      \phi = \pi/2 - \text{am}(\lambda T) ,
\]
where
\[
  T = \int_0^{\pi} \frac{du}{\sqrt{1-k^2\sin^2 u}}
\] 
is the period of revolution.

Then because the first chord is tangent to the envelope,
\[
     r = (R+a) \sin \phi
\]
And because the envelope belongs to the hyperbolic family generated by the motion,
\[
  \frac{4Ra}{(R+a)^2 - r^2} = k^2 .
\]

Substituting and some algebra leads to a quadratic equation in $a$ with roots
\[
  a =   R \sec \phi \left( \frac{2}{k^2} \sec \phi - \cos\phi \pm   \frac{2}{u} \sqrt{\frac{\sec^2\phi}{k^2} - 1}   \right) .
\]

We choose the minus sign because the plus sign gives the center of the circle in the upper half of the hyperbolic family (the circle that lies outside the pendulum circle.)



\begin{onlineOnly}
    \begin{center}
\desmos{svb8pwijet}{450}{600}  
\end{center}
\end{onlineOnly}

\href{https://www.desmos.com/calculator/svb8pwijet}{Jacobi Elliptic Functions and Theta Functions v 2}

But we should not ignore the other root. It gives the $y$-coordinate of the center of the circle in the other half of the hyperbolic family for an opposite motion with pendulum points starting at $(0,-R)$ and $(R\sin(2\phi),R\cos(2\phi))$. See the demo below.

\begin{onlineOnly}
    \begin{center}
\desmos{swzgtt3nup}{450}{600}  
\end{center}
\end{onlineOnly}

\href{https://www.desmos.com/calculator/swzgtt3nup}{Pendulum Motions Direct and Opposite}


\end{exploration}






\section*{Jacobi Closure Formulas}

The polygonal path closes when the time subtended by the chords is a rational multiple $\lambda$ of the period. The gives the euqation
\[
     \int_0^{\arccos(\frac{r}{R+a})} \frac{du}{(R+a)^2 - r^2 - 4a\sin^2(u/2)} = \lambda \int_0^{2\pi} \frac{du}{(R+a)^2 - r^2 - 4a\sin^2(u/2)} .
\]
Taking $R=1$ gives a relation between $a$ and $r$. We can use desmos to graph these curves as below. But it's hard to see how to use this to get a relation, even when $\lambda=1/3$ (triangle) or $\lambda  = 1/4$ (quadrilateral).


\begin{onlineOnly}
    \begin{center}
\desmos{wnef0tpllt}{450}{600}  
\end{center}
\end{onlineOnly}

\href{https://www.desmos.com/calculator/wnef0tpllt}{Jacobi Formula}



\section*{Revolving Pendulums}

We know the speed
\begin{align*}
    v_P &= \sqrt{v_0^2-4Rg \sin^2 (\theta/2)}  \\
           &=v_0 \sqrt{1-\frac{4Rg}{v_0^2} \sin^2 (\theta/2)} .
\end{align*}

Want to show that 
\[
    v_P = \frac{v_0}{R+a} \cdot \rho ,
\]
ie. that the speed is proportional to the distance $\rho = AP$ from $P$ to the focus. The constant of proportionality
\[
           \omega_0 = \frac{v_0}{R+a}
\]
is the rotation rate of chord $AP$ when it (the chord) is vertical.

Write the above product as
\begin{align*}
   \frac{v_0}{R+a} \cdot \rho  &= \frac{v_0}{R+a} \sqrt{R^2 + a^2 + 2aR \cos\theta} \\
                                         &= v_0 \sqrt{1 - \frac{4aR}{(R+a)^2} \sin^2 (\theta/2)} .
\end{align*}

This is all we need because remember that the focal distance $a$ was defined by the equality
\[
   \frac{4aR}{(R+a)^2} = \frac{4Rg}{v_0^2} .
\]

And since the speed of $P$ is also
\[
  v_P = \omega_P (\rho \sec\alpha)  = \omega_0 \rho , 
\]
where $\omega_P$ is the rotation rate of the chord and $\alpha$ is the angle between the radius $OP$ and segment $AP$, this tells us that
\begin{align*}
\omega_P &= \omega_0  \cos  \alpha  \\
                &= \omega_0 (R\cos \alpha) / R \\
                & = \left( \frac{\omega_0}{R} \right) \rho^\prime ,
\end{align*}
where $\rho^\prime$ is the length of chord $PP^\prime$. So the rotation rate of chord $PP^\prime$ is proportional to its length. The pendulum speed $v_P$ is proportional to $\rho = AP$.


\section{Coaxal Families}
One of my favorite precalculus problems is to find an equation of the line through the intersection points of two circles.


\section{Oscillating Pendulums}
This belongs earlier, but the focal points of the hypberbolic family generated by the pendulum motion with shape $k$ are the roots of the equation
\[
      \frac{4Rg}{(a+R)^2} = \frac{4Rg}{v_0^2} = k^2.
\]
The roots are
\[
      a = R \left(\frac{2}{k^2} -1  \pm \frac{2}{k}\sqrt{\frac{1}{k^2}-1} \right) . 
\]

With launch speed
\[
   v_0 < 2\sqrt{Rg}
\]
from stable equilibrium, the pendulum reaches a maximum angle $\theta_0$ where 
\[
    v_0^2 = 2Rg(1-\cos\theta_0) = 4Rg\sin^2 (\theta_0)/2 .
\]

The pendulum has no focal point (at least not the way we defined it for revolving pendulums) and the quadratic equation that defined the focal length 
\[
     g(a+R)^2 = v_0^2 a = 4Rag\sin^2 (\theta_0/2) .
\]
has complex roots
\[
     a = -R(\cos \theta_0 \pm i \sin \theta_0).
\]

We start by interpreting the expression
\[
       R^2 + a^2 + 2aR\cos\theta.
\]


It is still true, as for revolving pendulums, that
\[
  \rho = 
\]


\section*{Colliding Pendulums}
Instead of tracking the chords between two points moving in the same direction around a pendulum, send the points in opposite directions. The same logic shows that the lines between these points are tangent to a circle ${\cal C}_1$, now  in the outer half of the coaxal family generated by the motion. Let this circle have center $(0,a)$ and radius $r$.

Let one of the two collision points $P$ have displacement angle $\theta$ with coordinates $(R\sin\theta, -R \cos\theta)$. The tangent to the pendulum path at $P$ has equation
\[
       x\sin \theta - y\cos\theta - R = 0.
\]

The center $(0,a)$ and radius $r$ of ${\cal C}_1$ satisfy two conditions:

\pskip

(1) Since ${\cal C}_1$ is in the hyperbolic family generated by the motion with shape $k$,
\[
      4ar = k^2 \left(  (R+a)^2 - r^2   \right)
\]

\pskip

(2) Since the distance from $(0,a)$ to the tangent line at $P$ is $r$, 
\[
     \Big|  -y \cos \theta - R  \Big| = r .
\] 

These two conditions give a quadratic equation in $a$, with one solution $a=0$ (with corresponding radius $r=R$, ie the pendulum path). The other solution, the one of interest, is
\[
     a = \frac{4R}{k^2} \left( 1 - k^2 \sin^2 (\theta/2)    \right) ,
\]
with corresponding radius
\[
  r = R + a\cos \theta .
\]

The motions are illustrated below. For the same reasons as before, the rotation rate of the chord is proportional to its length and the speeds of the two motions are proportional to their respective distances to the points of tangency.

\begin{onlineOnly}
    \begin{center}
\desmos{5vecnkcwou}{450}{600}  
\end{center}
\end{onlineOnly}

\href{https://www.desmos.com/calculator/5vecnkcwou}{Revolving Pendulum Pairs of Points}


\section*{Asymptotic Pendulums}

\end{document}