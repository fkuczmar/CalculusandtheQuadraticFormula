\documentclass{ximera}
\title{The Doppler Effect}

\newcommand{\pskip}{\vskip 0.1 in}

\begin{document}
\begin{abstract}
This note is based on the Max Bondi's book \emph{Relativity and Common Sense}.
\end{abstract}
\maketitle


Doppler Effect with Sound:

\begin{onlineOnly}
    \begin{center}
\desmos{t1jmkomfwd}{800}{600}  
\end{center}
\end{onlineOnly}

\href{https://www.desmos.com/calculator/t1jmkomfwd}{Doppler Effect Stationary Observer With Sound 3}



\begin{onlineOnly}
    \begin{center}
\desmos{mygzufvdgk}{800}{600}         %zk06s3k6q4
\end{center}
\end{onlineOnly}

\href{https://www.desmos.com/calculator/zk06s3k6q4}{Doppler Stationary Observer 1}


The problem here should be to relate the frequencies at time $t=-0.5$ when $v=1$, $w=2$, and $h=4$. The result should be that 
\[
    f_1 = 0.7 f_2 . 
\]





\begin{onlineOnly}
    \begin{center}
\desmos{celviaogz7}{800}{600}  
\end{center}
\end{onlineOnly}

\href{https://www.desmos.com/calculator/celviaogz7}{Doppler Stationary Observer 2}

Doppler Effect with Sound:

\begin{onlineOnly}
    \begin{center}
\desmos{gxzmjpgkrr}{800}{600}  
\end{center}
\end{onlineOnly}

\href{https://www.desmos.com/calculator/gxzmjpgkrr}{Doppler Effect Stationary Observer 2 With Sound}

\begin{onlineOnly}
    \begin{center}
\desmos{sf3nnahyjf}{800}{600}  
\end{center}
\end{onlineOnly}

\href{https://www.desmos.com/calculator/sf3nnahyjf}{Doppler Effect Partial Derivatives}




\end{document}
