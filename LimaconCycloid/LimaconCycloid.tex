\documentclass{ximera}
\title{Limacons and the Cycloid, Arclength}

\newcommand{\pskip}{\vskip 0.1 in}

\begin{document}
\begin{abstract}
Relationship between their arclengths.
\end{abstract}
\maketitle


\begin{onlineOnly}
    \begin{center}
\desmos{bpvuhnhi35}{900}{600}
\end{center}
\end{onlineOnly}

\href{https://www.desmos.com/calculator/bpvuhnhi35}{163: Limacon Cylcoid Arclength 2}

As a circle ${\cal C}$ of radius $a$ rolls on the $x$-axis, a fixed point in the reference frame of ${\cal C}$ and $b>a$ units from its center sweeps out the limacon of Figure 1.
 
A fixed point on the same circle ${\cal C}$ traces the cycloid in Figure 1.

Let $s_1$ be the arclength of the portion of the limacon above the $x$-axis between inclination angles $0$ and $\phi$. Let $s_2$ be the arclength of the portion of the limacon below the axis between the same inclination angles. And let $s_3$ be the arclength of the cycoid between these inclination angles.


Theorem: $s_1 - s_2 = s_3$.

Our proof of the theorem is geometric.

\begin{onlineOnly}
    \begin{center}
\desmos{hckgktfx4p}{900}{600}
\end{center}
\end{onlineOnly}

\href{https://www.desmos.com/calculator/hckgktfx4p}{163: Limacon Cylcoid Arclength 3}



\end{document}
