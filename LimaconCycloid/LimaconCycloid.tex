\documentclass{ximera}
\title{Limacons and the Cycloid, Arclength}

\newcommand{\pskip}{\vskip 0.1 in}

\newtheorem*{theorem9}{Theorem}

\begin{document}
\begin{abstract}
Relationship between their arclengths.
\end{abstract}
\maketitle
 

In a recent article, ?? made the surprising observation that the length of the trochoid
\[
    (x,y) = (a\theta + b\sin\theta , a + b\cos\theta) \, , \, 0\leq \theta \leq 2\pi ,
\]
above the $x$-axis exceeds the length below by an amount equal to the length $8a$ of the cycloid
\[
    (x,y) = ( a\theta + a\sin\theta , a + a\cos\theta ) \, , \, 0\leq \theta \leq 2\pi .
\]

\begin{onlineOnly}
    \begin{center}
\desmos{6de49gy5g8}{900}{600}
\end{center}
\end{onlineOnly}

\href{https://www.desmos.com/calculator/6de49gy5g8}{163: Limacon Cylcoid Arclength 1}

%The surprise is that that difference is independent of the size of the loop.

The purpose of this note is to give a geometric proof of this fact.

A first step in this direction was to look for a more general equality of arclengths by associating each point of the limacon above the $x$-axis with a point below. One possibility is to pair points with same inclination angle (the angle the tangent to the limacon makes with the horizontal). Much to my delight, this appeared to work. That is, the arcle





\begin{onlineOnly}
    \begin{center}
\desmos{bpvuhnhi35}{900}{600}
\end{center}
\end{onlineOnly}

\href{https://www.desmos.com/calculator/bpvuhnhi35}{163: Limacon Cylcoid Arclength 2}

As a circle ${\cal C}$ of radius $a$ rolls along the $x$-axis, a fixed point in the reference frame of ${\cal C}$ that is $b>a$ units from ${\cal C}$'s center sweeps out the limacon of Figure 1.
 
A fixed point \emph{on} the same circle of radius $a$ traces the cycloid in Figure 1.

Let $s_1$ be the arclength $A_1P_1$ of the limacon above the $x$-axis between inclination angles $0$ and $\phi$. Let $s_2$ be the arclength $A_2P_2$ of the limacon below the axis between the same inclination angles. And let $s_3$ be the arclength $A_3P_3$ of the cycoid between these inclination angles.

\begin{theorem9}
With the above notation, $s_1 - s_2 = s_3$.
\end{theorem9}

\begin{proof}

Imagine three copies of ${\cal C}$ rolling along the $x$-axis as illustrated below.

\begin{onlineOnly}
    \begin{center}
\desmos{hckgktfx4p}{900}{600}
\end{center}
\end{onlineOnly}

\href{https://www.desmos.com/calculator/hckgktfx4p}{163: Limacon Cylcoid Arclength 3}

\end{proof}

\end{document}
