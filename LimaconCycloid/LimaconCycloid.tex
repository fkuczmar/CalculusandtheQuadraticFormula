\documentclass{ximera}
\title{An Arclength Theorem for Cycloids}

\newcommand{\pskip}{\vskip 0.1 in}

\newtheorem{theorem9}{The Arclength Theorem}

\begin{document}
\begin{abstract}
Relationship between their arclengths.
\end{abstract}
\maketitle
 

In a recent article, D. Robinson made the intriguing observation that the length of the trochoid
\begin{equation} 
    (x,y) = (a\theta + b\sin\theta , a + b\cos\theta) \, , \, -\pi\leq \theta \leq \pi , \label{Eq:Limacon}
\end{equation}
above the $x$-axis exceeds its length below by an amount equal to the length $8a$ of the cycloid
\begin{equation}
    (x,y) = ( a\theta + a\sin\theta , a + a\cos\theta ) \, , \, -\pi\leq \theta \leq \pi . \label{Eq:Cycloid}
\end{equation}

\begin{onlineOnly}
    \begin{center}
\desmos{ty0hdz7uyj}{900}{600}
\end{center}
\end{onlineOnly}

\href{https://www.desmos.com/calculator/ty0hdz7uyj}{Limacon Cylcoid Arclength 1B}

%The surprise is that that difference is independent of the size of the loop.

The purpose of this note is to give a geometric proof of this fact and view some of Robinson's other observations in a broader context.

Our first step in this direction is to look for a more general equality of arclengths by associating each point of the limacon above the $x$-axis with a point below. It seems like the only choice was to pair points with same inclination angle (the angle a curve makes with the horizontal) and experiments with desmos suggest that the same equality of arclengths holds between any two inclination angles.

In the figure below, for example, the difference $s_1 - s_2$ in the arclengths $A_1P_1$ and $A_2P_2$ in the limacon between indclination angles $0$ and $\phi$ equals the arclength $s_3$ of the cycloid between these same angles. 

\begin{onlineOnly}
    \begin{center}
\desmos{0vodc6gvp5}{900}{600}   %bpvuhnhi35
\end{center}
\end{onlineOnly}

\href{https://www.desmos.com/calculator/0vodc6gvp5}{Limacon Cylcoid Arclength 2B}


The figure below makes the equality of arclengths plausible if you focus on the short arcs of the same color between the same inclination angles.

\begin{onlineOnly}
    \begin{center}
\desmos{4xn0xtuszi}{900}{600}   
\end{center}
\end{onlineOnly}

\href{https://www.desmos.com/calculator/4xn0xtuszi}{Cylcoid Trochoid Color}

Trying to prove this more general result works to our advantage as its shifts our focus from a global one (proving an equality of integrals) to  a local one (proving an equality of \emph{differential} arclengths). %This worked to our advantage.

\begin{theorem9}
With $s_1$, $s_2$ the respective arclengths of the limacon (\ref{Eq:Limacon}) above and below the $x$-axis between inclination angles $0$, and $\phi$ and $s_3$ the arclength of the cycloid (\ref{Eq:Cycloid}) between these same angles,
\[
s_1 - s_2 = s_3.
\]
\end{theorem9}

Our proof of this theorem relies partly on the instantaneous center of rotation. Readers familiar with this idea should feel free to skip the next section and move directly to the proof.

\section{Preliminaries}
It might seem counterintuitive that at any moment a rolling wheel is rotating \emph{not} about its center but rather about its point of contact with the ground. This statement has two implications for a point $P$ at rest in the reference frame of a rolling wheel, like a pebble stuck in a tire.

\begin{enumerate}
\item The speed of $P$ is equal to the product $\rho \omega$ of the distance $PQ$ from $P$ to the point of contact and the rotation rate of the wheel.

\item The velocity of $P$ is orthogonal to segment $\overline{PQ}$.
\end{enumerate}

%The photograph of a rolling wheel below gives some sense of (a); points on the spokes farther from the point of contact are moving faster and the spokes blur....

The figure below shows a cycloid, the curve traced by a point $P$ fixed on the circumference of a rolling wheel. The marked angle $\theta$ is the wheel's angle of rotation measured from an arbitrary starting point, in this case when $\overline{QP}$ is vertical and $P$ coincides with $A$. Because of (b) above, the incination of the cycloid at $P$ (ie. the angle between the tangent and the horizontal) is equal to the measure of $\angle BQP$ (ie. the angle between the normal $\overline{PQ}$ and the vertical).

The points on the cycloid are spaced at equal intervals of rotation and suggest that the speed $v= \rho (d\theta/dt)$ of the tracing point is proportional to its distance from the center of rotation.  

\begin{onlineOnly}
    \begin{center}
\desmos{ilinb0qn7q}{900}{600}   %bpvuhnhi35
\end{center}
\end{onlineOnly}

\href{https://www.desmos.com/calculator/ilinb0qn7q}{Center of Rotation}

To prove the arclength theorem, it will be helpful to think differentially instead of dynamically. Then (a) is equivalent to saying that as the wheel turns through the differential angle $d\theta>0$, the tracing point traverses the differential distance $ds = \rho \, d\theta$. 

For the cycloid above, for example, generated by a circle of radius $a$,
\begin{align*}
     ds = &\rho \, d\theta \\
             &= 2a \cos\phi\, d\theta \\
              &= 2a \cos(\theta/2)\, d\theta .  
\end{align*}

{\bf Remark:} Integrating these differential arclengths shows the length of the cycloid from $A$ to $P$ is
\[
       s = 4a\sin(\theta/2) = 4a \sin\phi %2 \text BP .
\]
and equal to twice the length of chord $\overline{BP}$.

But my favorite way to prove this equality is by comparing the differential arclength $ds = 2a\cos\phi\, d\theta$ with the differential change $dc$ in the distance $c = BP$. In the reference frame of the rolling circle where $P$ is at rest, $B$ traverses the differential arclength $a\, d\theta$ as the circle turns through the angle $d\theta$. But since $B$ moves horizontally in the direction inclined at the angle $\phi$ to $\overrightarrow{PB}$ %(and not directly away from $P$), 
\[
    dc = (a\, d\theta) \cos \phi = ds/2.
\]





\section{A Proof}

\begin{proof}
To prove the arclength theorem, imagine three circles ${\cal C}_i$, $i=1,2,3$, rolling along the $x$-axis as the points $P_i$ (at rest in the reference frames of these circles) respectively sweep out the upper limacon, lower limacon, and cycloid.  We suppose also as illustrated in Figure 3, that the segments $\overline{Q_iP_i}$ from the centers of rotation to the tracing points remain parallel at all times.

%We suppose the circles start with the points $P_i$ at the same inclination angle $\phi=0$ and the segments $Q_iP_i$ vertical. And we also assume as illustrated in Figure 3, the segments $\overline{Q_iP_i}$ from the centers of rotation to the tracing points remain parallel at all times.

Our goal is to show
\begin{equation} \label{Eq:DiffArcs}
      ds_1 - ds_2 = ds_3 ,
\end{equation}
where  $ds_i$ are the differential arclengths along the curves between inclination angles $\phi$ and $\phi + d\phi$. 

\begin{onlineOnly}
    \begin{center}
\desmos{qtse3eivit}{900}{600}   %snq3rgm0a4
\end{center}
\end{onlineOnly}

\href{https://www.desmos.com/calculator/qtse3eivit}{Limacon Cylcoid Arclength 3C}

These differential arclengths are at distances $\rho_i = Q_iP_i$ from the respective centers of rotation $Q_1$, $Q_2$, $Q_3$. A key point in what follows is that %by the symmetry of the first (red) circle about the diameter perpendicular to $\overline{P_1R_1}$,
\[
   \rho_2 = Q_2P_2 = O_1R_1 = S_1P_1 ,
\]
where the last equality follows from the symmetry of ${\cal C}_1$ about the diameter perpendicular to $\overline{P_1R_1}$,
This equality has three consequences:

\begin{enumerate}
\item  $\rho_1 - \rho_2 = \rho_3$ ,

\item $\theta_1 + \theta_2 = \theta_3$, where the rotation angles $\theta_1$ and $\theta_3$ are measured from the upward vertical and $\theta_2$ from the downward vertical as illustrated above, and %the rotation angles are measured from the vertical,  as indicated above) , and

\item $\frac{d\theta_1}{d\theta_2} = \frac{\rho_1}{\rho_2}$.
\end{enumerate}

The first is immediate, but it would be a misktake (one that I made) to think this equality of distances directly implies the equality of differential arclengths (\ref{Eq:DiffArcs}). The reason is that the circles turn through unequal differential angles $d\theta_i$ in sweeping out these arclengths. We also need (b) and (c) to prove (\ref{Eq:DiffArcs}), which we now write in the form %we need to know something about the relationships among the rotation angles and their differential changes.  % $ds_i$, this equality does \emph{not} directly imply the equality of differential arclengths (\ref{Eq:DiffArcs}). We need to know more to show (\ref{Eq:DiffArc}), which we write now in the form
\[
       \rho_1 \, d\theta_1 - \rho_2 \, d\theta_2 = \rho_3 \, d\theta_3 .
\]
%we need to establish some relationships among the rotation angles and their differential changes.


%because the circles turn through unequal differential angles $d\theta_i$ in sweeping out the arcs. %We can see this globally since in sweeping out the full curves (upper limacon, lower limacon, and cycloid) the circles turn through the unequal angles
%\[
 %     (\Delta \theta_1 = 2\pi) > (\Delta \theta_1 = 2\arccos(-a/b)) > (\Delta \theta_3 = 2\pi - 2\arccos(-a/b) ) . 
%\]
%But since the circles in Figure 3 turn through unequal differential angles in sweeping out the differential arclengths because in sweeping out the full curves (upper, lower limacon, and cycloid) the circles turn through the unequal angles 
%\[
 %     (\Delta \theta_1 = 2\pi) > (\Delta \theta_1 = 2\arccos(-a/b)) > (\Delta \theta_3 = 2\pi - 2\arccos(-a/b) ) . 
%\]
%So we need to find some relationship among these differential arclengths to show the equality (\ref{Eq:DiffArc}), which we write now in the form
%\[
%       \rho_1 \, d\theta_1 - \rho_2 \, d\theta_2 = \rho_3 \, d\theta_3 .
%\]



%Figure 3 shows three copies of a circle ${\cal C}$ of radius $a$ rolling along the $x$-axis. The top two show points $P_1$ and $P_2$ (each attached to a circle of radius $b$ rolling with ${\cal C}$) sweeping out the upper and lower arcs of the limacon as the circles turn through the respective angles $\theta_1$ (measured from the upward-pointing vertical) and $\theta_2$ (measured from the downward vertical). The lower shows the corresponding arc of the cycloid traced by $P_3$ as ${\cal C}$ turns though the angle $\theta_3$.

%%show points $P the upper and lower arcs of the limacon traced by the respective points $P_1$ and $P_2$ (each attached to a circle of radius $b$ rolling with ${\cal C}$) as ${\cal C}$ turns through the respective angles $\theta_1$ (measured from the upward pointing vertical) and $\theta_2$ (measured from the downward-pointing vertical). The lower shows the corresponding arc of the cycloid traced by $P_3$ as ${\cal C}$ turns though the angle $\theta_3$.




%We regard the corresponding arclengths $s_1$, $s_2$, $s_3$ and rotation angles $\theta_i$ as functions of the common inclination angle $\phi$ of the curves at $P_1$, $P_2$, and $P_3$. Our aim is to relate the differential arclengths $ds_i$ and the differential angles of rotation $\theta_i$ to the differential change $d\phi$ in the inclination angle.

%Just two additional facts suffice:   %It turns out that we just need  two facts:
%\begin{enumerate}
% \item $ds_i = \rho_i \, d\theta_i$ ,

%\item $\theta_1 + \theta_2 = \theta_3$, where the rotation angles $\theta_1$ and $\theta_3$ are measured from the upward vertical and $\theta_2$ from the downward vertical as illustrated above, and %the rotation angles are measured from the vertical,  as indicated above) , and

%\item $\frac{d\theta_1}{d\theta_2} = \frac{\rho_1}{\rho_2}$.
%\end{enumerate}

Assuming (b) and (c) for the moment, then %These are enough to prove Theorem 1. For then
\begin{align*}
     ds_1 - ds_2 &= \rho_1 \, d\theta_1 - \rho_2 \, d\theta_2 \\
                       &= (\rho_3 + \rho_2) \, d\theta_1 - (\rho_1 - \rho_3)\, d\theta_2 \\
                       &= \rho_3 (d\theta_1 + d\theta_2) + (\rho_2 \, d\theta_1 - \rho_1\, d\theta_2) \\
                       &= \rho_3 \, d\theta_3 \\
                       &= ds_3 .
\end{align*}

It remains to prove (b) and (c).

The proof of (b) follows from the composite figure below showing the three rotation angles $\theta_i$ and the inclination angle $\phi$. Since $\overline{S_1P_1}\cong \overline{Q_1R_1}$, we know $\Delta C_1S_1P_1\cong \Delta C_1Q_1R_1$. Therefore, $m\angle P_1C_1S_1 = \theta_2$ and 
\[
 \theta_3 = m \angle B_1C_1S_1 = \theta_1 + \theta_2.
\]


\begin{onlineOnly}
    \begin{center}
\desmos{txbihum5r3}{900}{600}   
\end{center}
\end{onlineOnly}

\href{https://www.desmos.com/calculator/txbihum5r3}{Cylcoids Proof of Theorem 1}

To prove (c), we borrow an idea from Tabachnikov's \emph{Geometry and Billiards} and consider the differential changes $b\, d\theta_i$, $i = 1,2$  in the respective arclengths $b\theta_i$ as the the chord $\overline{P_1R_1}$ turns through the differential angle $d\phi$.  

\begin{onlineOnly}
    \begin{center}
\desmos{i2050p22ud}{900}{600}      %xvwr9mhuvw
\end{center}
\end{onlineOnly}

\href{https://www.desmos.com/calculator/i2050p22ud}{Cylcoids Proof of Theorem 1BB}

%i2050p22ud

In the figure above, the differential arcs $ZP_1^\prime$ and $WR_1^\prime$ (circular arcs centered at $Q_1$) have lengths
\[
    ZP_1^\prime = \rho_1 \, d\phi
\]
and
\[
  WR_1^\prime =  \rho_2 \, d\phi.
\]

But since a chord makes congruent angles with a circle at its endpoints, the differential right triangles $\Delta P_1ZP_1^\prime$ and $\Delta R_1 W R_1^\prime$ are similar. Hence,
\[
      \frac{P_1P_1^\prime}{R_1 R_1^\prime}  =  \frac{ZP_1^\prime}{WR_1^\prime} 
\]
and
\[
    \frac{b\, d\theta_1}{b \, d\theta_2} = \frac{ \rho_1 \, d\phi}{ \rho_2 \, d\phi} .
\]
%and (b) follows.
 
\end{proof}

{\bf Remark:} Writing (\ref{Eq:DiffArcs}) as
\[
   \frac{ds_1}{d\phi} - \frac{ds_2}{d\phi} = \frac{ds_3}{d\phi}
\]
gives another interpretation of the Arclength theorem as a relationship
\[
     r_1 - r_2 = r_3
\]
among the radii of curvature at the same inclination angle.





\section{A Generalization}
A classic problem asks how many rotations one quarter makes as it rolls once around the outside of another. 

Compared with rolling along a line, the rolling quarter picks up an extra rotation and makes two rotations as it rolls once through its circumference.

\begin{onlineOnly}
    \begin{center}
\desmos{hp5fdl6k6d}{900}{600}
\end{center}
\end{onlineOnly}

\href{https://www.desmos.com/calculator/hp5fdl6k6d}{Rolling Quarter 2}

The same is true locally. As the quarter turns the differential angle $d\theta$ about the center of the other, it rotates through the differential angle $d\theta^* = 2\, d\theta$. 

More generally, as a circle ${\cal C}$ of radius $a$ rolls through the differential arclength $d\sigma$ of its circumference around the outside of a fixed circle ${\cal C}_r$ with radius $r$, it turns through the angle $d\theta = d\sigma/a$ relative to the common tangent line rolling with it. At the same time, the tangent turns with the same sense through the angle $d\theta^\prime = d\sigma/r$ about the center of the fixed circle. So ${\cal C}$ turns through the angle
\begin{align*}
     d\theta^* &= \left(\frac{1}{a} + \frac{1}{r} \right) \, d\sigma \\
                    &= \left( 1 + \frac{a}{r}  \right) \, d\theta . 
\end{align*}
relative to ${\cal C}_r$.

These relationships imply the arclength of a cardioid is twice that of the corresponding cycloid. The cardioid is traced by a point attached to a circle as it rolls around another circle of the same radius, the cycloid by a point on a circle of the same radius roling on a line.

For imagine tracing points $P$ and $P^*$ attached to these circles of radius $a$, the first rolling along a line, the second around a fixed circle (Figure ?? below). Imagine also the circles starting with their tracing points diameterically opposite the points of tangency. Now let the circles roll through the arclength $d\sigma$ of their circumferences. Then the first turns through the angle $d\theta = d\sigma/a$ and the second through the angle $d\theta^* = 2\, d\theta$. Since the distances $\rho = QP$ and $\rho^* = Q^*P^*$ from the centers of rotation to the tracing points are equal, the point $P^*$ sweeps out an arclength
\[
      ds^* = QP\, d\theta^* = 2\,ds
\]
equal to twice that traversed by $P$.

%For suppose the circle rolls through the arclength $d\sigma$ of its circumference as it turns through the angle $d\theta = d\sigma / a$ about the center of the fixed circle. Then it rotates through the angle $d\theta^* = 2d\theta$, twice the rotation angle of a circle with radius $a$ rolling on a line through the same differential arclength. So with equal distances $Q_1P_1$ and $Q_2P_2$ of the tracing points from the centers of rotations, 

%The figure below compares the arclengths of a cyloid and cardioid, each generated by circles of the same radius. The two tracing points $P_1$ and $P_2$ start in the same position

\begin{onlineOnly}
    \begin{center}
\desmos{grcnlkuqan}{900}{600}
\end{center}
\end{onlineOnly}

\href{https://www.desmos.com/calculator/grcnlkuqan}{Rolling Quarter and Cardioid 2}

There is nothing special about the points $P$, $P^*$ being on the circles. 


\begin{onlineOnly}
    \begin{center}
\desmos{nfg2ofyt1t}{900}{600}
\end{center}
\end{onlineOnly}

\href{https://www.desmos.com/calculator/nfg2ofyt1t}{Wrapping cycloids around circles General 2}



\section{Limacons}

\begin{onlineOnly}
    \begin{center}
\desmos{kuhagddz3r}{900}{600}
\end{center}
\end{onlineOnly}

\href{https://www.desmos.com/calculator/kuhagddz3r}{Wrapping Cycloids around Circles Limacon}







\section{Nephroid}

\begin{onlineOnly}
    \begin{center}
\desmos{czhdwjarf3}{900}{600}
\end{center}
\end{onlineOnly}

\href{https://www.desmos.com/calculator/czhdwjarf3}{Wrapping Cycloids around Circles Nephroid}


%\begin{onlineOnly}
 %   \begin{center}
%\desmos{hw0lkueztk}{900}{600}
%\end{center}
%\end{onlineOnly}

%\href{https://www.desmos.com/calculator/hw0lkueztk}{Limacon Cycloid Arclength 3}



%\begin{onlineOnly}
%    \begin{center}
%\desmos{sgpsbfqdze}{900}{600}
%\end{center}
%\end{onlineOnly}

%\href{https://www.desmos.com/calculator/sgpsbfqdze}{Epicycloids 4 Neprhoid}

%\begin{onlineOnly}
%    \begin{center}
%\desmos{v9bsdhviwh}{900}{600}
%\end{center}
%\end{onlineOnly}

%\href{https://www.desmos.com/calculator/v9bsdhviwh}{Epicycloids 4 Neprhoid 2}




\section{Ellipses}

\begin{onlineOnly}
    \begin{center}
\desmos{yzuot0hbvv}{900}{600}
\end{center}
\end{onlineOnly}

\href{https://www.desmos.com/calculator/yzuot0hbvv}{Epicycloids 3 Ellipse}

yzuot0hbvv


\begin{onlineOnly}
    \begin{center}
\desmos{yillqxp8mg}{900}{600}   %yfoczp8y2g
\end{center}
\end{onlineOnly}

\href{https://www.desmos.com/calculator/yillqxp8mg}{Wrapping Cycloids Around Circles Ellipse 2B}

yillqxp8mg


\end{document}
