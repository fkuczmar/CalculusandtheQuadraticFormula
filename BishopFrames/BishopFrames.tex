\documentclass{ximera}
\title{Bishop Frames}

\newcommand{\pskip}{\vskip 0.1 in}

\begin{document}
\begin{abstract}
This note is based on the paper \emph{There is more than one way to frame a curve}.
\end{abstract}
\maketitle


\section*{Introduction}
The key idea of the paper is captured in the first two sentences of Section 1:

\pskip

\emph{Relatively parallel fields.} We say that a normal vector field $M$ along a curve is \emph{relatively parallel} if its derivative is tangential. Such a field turns only whatever amount is necessary for it to remain normal, so it is as close to being parallel without losing normality.

\pskip

For me it helps to have a physical model or to at least imagine one. Start by bending a wire into a space curve. Then cut out a small cardboard disk and punch a hole is its center. Now \emph{slide} the disk along the wire, keeping the disk perpendicular to the curve at all times \emph{without} turning the disk about the wire.

It's the without ``turning part'' that's crucial. This is equivalent to requiring each point of the disk to move in the direction of the wire (ie. normal to the disk) at all times. Since there are infinitely many ways to slide the wire normally along the wire with rotation and only one way without, we'll start with an example of the first.


\begin{exploration}
The animation below is generated by the Frenet frame (principal normal (green) and binormal (orange). The eight points are fixed relative to this frame. The disk should appear to rotate about the cylindrical helix; the paths traced by the points are \emph{not} normal to the disk. You can take control of the slider $w$ in Line 32.

\begin{onlineOnly}
    \begin{center}
\desmosThreeD{vroc5lhox8}{800}{600}  
\end{center}
\end{onlineOnly}

\href{https://www.desmos.com/3d/vroc5lhox8}{163: Tube around cylindrical helix 4}
\end{exploration}





\begin{exploration}
The animation below is generated by a Bishop frame frame (not shown). The eight points are fixed relative to this frame, but the principal normal (green) rotatates relative to this frame. The disk should appear to move along the cylindrical helix \emph{without} rotating; the paths traced by the points are normal to the disk. The vectors (not shown) from the disk's center to the eight points are relatively parallel to the helix. You can take control of the slider $w$ in Line 32.

\begin{onlineOnly}
    \begin{center}
\desmosThreeD{fg6l58x6tl}{800}{600}  
\end{center}
\end{onlineOnly}

\href{https://www.desmos.com/3d/fg6l58x6tl}{163: Tube around cylindrical helix 3}

\end{exploration}

\begin{exploration}
This animation starts with a tube around the cylindrical helix and a net on this tube generated by the Frenet frame. Note that the net is not orthogonal. Move the slider $p$ in Line 47 to untwist the net and transform it into an orthogonal net generated by a Bishop frame.

\begin{onlineOnly}
    \begin{center}
\desmosThreeD{agzievbron}{800}{600}  
\end{center}
\end{onlineOnly}

\href{https://www.desmos.com/3d/agzievbron}{163: Tube around cylindrical helix 2}
\end{exploration}

We've seen this kind of thing before when we studied geodesic curvature, where the idea was to construct a parallel vector field along a curve lying on a surface. The field is tangent to the surface and the derivative of the field is normal to the surface. Such a field looks constant to the residents of the surface because its derivative is invisible. Or put another way, the field turns only whatever amount is necessary for it to remain tangent to the surface. The geodesic curvature is the rate (relative to arclength) at which the unit tangent turns relative to a parallel field.

Both parallel and relatively parallel fields can be generated by a bicycle wheel......

\begin{exploration}
\href{https://www.desmos.com/3d/frbiiccyfd}{Parallel vector field along a circle of latitude on a sphere}
\end{exploration}




\section*{Computing Relatively Parallel Fields}
%The image of a disk sliding along a curve without rotating

The key to computing a relatively parallel vector field along a curve is to destroy the rotation of the Frenet frame about its tangent vector $\overrightarrow{T}$ by replacing the principal normal with a normal vector relatively parallel to the curve. To do this we need to know the component of the Frenet frame's rotation about $\overrightarrow{T}$.

Thes rotational components of the Frenet frame about the unit tangent $\overrightarrow{T}$, the principal normal $\overrightarrow{N}$, and the binormal $\overrightarrow{B}$ are governed by the Frenet equations. 

The first equation (using primes for differentiation with respect to arclength)
\begin{equation}
       \overrightarrow{T}^\prime = \kappa \overrightarrow{N} ,   \label{Eq:Curvature}
\end{equation}
is really just the definition of curvature $\kappa \geq 0$ and the principal normal. From this it follows that $\overrightarrow{B}^\prime$ is also a multiple (not necessarily positive) of $\overrightarrow{B}$. 

\begin{exercise}  \label{E:Kldsfttdd}
(a) Prove algebraically that
\begin{equation}
   \overrightarrow{B}^\prime = - \tau \overrightarrow{N}   \label{Eq:Torsion}
\end{equation}
for some scalar $\tau$. 

\begin{hint}
Differentiate both sides of the equation
\[
  \overrightarrow{B} \cdot \overrightarrow{T} = 0
\]
with respect to arclength.
\end{hint}

(b) Explain geometrically why $\overrightarrow{B}^\prime$ is a scalar multiple of $\overrightarrow{B}$. 
\end{exercise}

The scalar $\tau$ (a function of the arclength parameter) in equation (\ref{Eq:Torsion}) is the \emph{torsion} of our curve. It measures the rate at which the binormal (and hence the osculating plane) rotates with repsect to arclength. Unlike curvature, torsion may be negative. The negative sign in (\ref{Eq:Torsion}) is arbitrary. It might there to givee a right-handed helix 
postivie torsion.

\begin{exercise} \label{E:dsft4etdg}
Explain with a picture why a right-handed cylindrical helix has positive torsion.
\end{exercise}

\begin{exercise}   \label{QQ:Podsfgttga}
(a) Use the first two Frenet equations to find the third,
\[
     \overrightarrow{N}^\prime = \answer{-\kappa} \overrightarrow{T} + \answer{\tau} \overrightarrow{B} .
\]
\begin{hint}
Differentiate each of the equations
\[
 \overrightarrow{N}\cdot \overrightarrow{B} = \overrightarrow{N}\cdot\overrightarrow{T} = 0
\]
with respect to arclength.
\end{hint}

(b) Explain the third Frenet equation geometrically.
\end{exercise}


This is all we need to describe a relatively parallel vector field $\overrightarrow{M}$ along a curve by finding a function $\theta = f(s)$ of the arclength parameter to make
\begin{equation}
          \overrightarrow{M} = \cos\theta \overrightarrow{N} \cos \theta + \sin\theta \overrightarrow{B} ,   \label{Eq:Bishop}
\end{equation}
relatively parallel.

If we keep in mind the idea of destroying the rotational component of the Frenet frame about $\overrightarrow{T}$, it is possible to immediately describe all such functions $f$. But if this is not clear, we can follow Bishop and take a more computational approach. 

\begin{exercise}  \label{E:dfstr35tr43}
Differentiate both sides of (\ref{Eq:Bishop}) with respect to arclength and use the Frenet equations to express $\theta$ as a function of $s$.

\begin{hint}
Keep in mind the condition for $\overrightarrow{M}$ to be relatively parallel. 
\end{hint}
\end{exercise}

\begin{exercise}  \label{E:MDnfddd}
Parameterize the relatively parallel vector fields along a counterclockwise cyllindrical helix of radius $a$ that makes the angle $\pi/2 - \phi$ with the cylinder's axis of symmetry.
\end{exercise}




\section*{Roller Coaster Tracks}
Pick a space curve for one track of a roller coaster. How would we design the other track?

The tracks should have the following properties:

a) Let the normal plane to Track A at a point $P$ intersect Track B in point $Q$. Then $\overline{AB}$ should be perpendicular to Track B at $Q$.

(b) The tracks should be parallel at $P$ and $Q$.

\pskip

So from (a), if ${\bf r}(s)_A$ is an arclength paramerization of Track A, then Track B has parameterization
\[
      {\bf r}_B = {\bf r}_A + L ( {\bf N} \cos\theta  + {\bf B} \sin\theta ) ,
\]
where $\theta = f(s)$ is a function (to be determined) of the the arclength parameter along Track A. To parameterize Track B in terms of $s$, we need to find an expression for $f(s)$.
 
Differentiating with repsect to $s$ gives
\begin{align*}
      {\bf r}_B^\prime &= {\bf T} -L\theta^\prime \sin\theta {\bf N} + L\cos\theta (-\kappa {\bf T} + \tau {\bf B}) + L\theta^\prime\cos\theta {\bf B} -L\tau \sin\theta {\bf N}   \\
                                 &= (1-\kappa L){\bf T} - L\sin\theta(\tau +\theta^\prime){\bf N} + L \cos\theta (\theta^\prime + \tau){\bf B}
\end{align*}
The requirement that ${\bf r}_B^\prime$ be parallel to ${\bf T}$ (ie. that the tracks be parallel at each pair of corresponding points $P$ and $Q$) is satisfied if and only if
\[
  \theta^\prime = \frac{d\theta}{ds} = -\tau(s)
\]
Taking $\theta = 0$ when $s=0$ (ie. supposing the Bishop frame coincides with the Frenet frame at $s=0$), we get
\[
        \theta = - \int_0^s \tau(s^*) ds^* . 
\]
This makes inutuitive sense. As Trevor pointed out, torsion is the rotation rate of ${\bf B}$ around ${\bf T}$ and the Bishop frame backs out (kills) the torsion in the sense that the frame does not rotate about ${\bf T}$.

\pskip

\begin{exploration}
This animation starts with a cyclindrical helix (purple) for Track A and constructs a partner Track B.

\begin{onlineOnly}
    \begin{center}
\desmosThreeD{rxaisxokzj}{800}{600}  
\end{center}
\end{onlineOnly}

\href{https://www.desmos.com/3d/rxaisxokzj}{163: Roller Coaster Tracks}
\end{exploration}



Trevor: The torsion is the rotation rate of ${\bf B}$ around ${\bf T}$.

Michael: Do the two tracks have the same normal and binormals?

Distinction between differentiating with respect to time and with respect to the arclength parameters along each track.

Serge Lang Basic Mathematics

\pskip





\section*{Rotational Motion of Frenet and Bishop Frames}


\begin{exploration}
The animation below shows the motion of the Frenet frame, traveling (top) and pinned (bottom) for a cylindrical helix with radius $a$ and pitch angle $\phi$. What is the frame's centrode (ie. angular velocity vector with respect to arc length)?

\begin{onlineOnly}
    \begin{center}
\desmosThreeD{hkqr8xjops}{800}{600}  
\end{center}
\end{onlineOnly}

\href{https://www.desmos.com/3d/hkqr8xjops}{163: Rotational Motion of Frenet Frame}
\end{exploration}




\begin{exploration}
The animation below shows the motion of a Bishop frame, traveling (top) for a cylindrical helix with radius $a$ and pitch angle $\phi$. Can you predict the motion of the pinned frame (bottom)? What is the frame's centrode?
\begin{onlineOnly}
    \begin{center}
\desmosThreeD{hkqr8xjops}{800}{600}  
\end{center}
\end{onlineOnly}

\href{https://www.desmos.com/3d/hkqr8xjops}{163: Rotational Motion of Frenet Frame}
\end{exploration}


For the Frenet frame, since
\[
      {\bf B}^\prime = -\tau {\bf N} ,
\]
\[
   {\bf T}^\prime = \kappa {\bf N},
\]
and
\[
   {\bf N}^\prime = \tau {\bf B} -\kappa {\bf T}  ,
\]
its angular velocity vector is 
\[
  \boldsymbol{\Omega}_f =   \kappa {\bf B} + \tau {\bf T}  .
\]
As a check, 
\[
    {\bf T}^\prime =    \boldsymbol{\Omega}_f \times {\bf T} = \kappa {\bf B} \times {\bf T} = \kappa {\bf N} 
\]
and 
\[
    {\bf B}^\prime =    \boldsymbol{\Omega}_f \times {\bf B} = \tau {\bf T} \times {\bf B} = - \tau {\bf N} 
\]
and
\[
    {\bf N}^\prime =    \boldsymbol{\Omega}_f \times {\bf N} =  (\tau {\bf T} + \kappa {\bf B})\times {\bf N} = \tau {\bf B}-\kappa{\bf T} .
\]

What about the angular velocity of a Bishop frame $({\bf T}, {\bf M}_1, {\bf M}_2)$? Following Bishop's notation, since by definition
\[
     {\bf M}_1^\prime = -k_1 {\bf T} , 
\]
and
\[
   {\bf M}_2^\prime = -k_2 {\bf T} , 
\]
it follows that
\[
   {\bf T}^\prime  = k_1 {\bf M}_1 + k_2 {\bf M}_2 = \kappa {\bf N} .
\]
So the angular velocity of the frame is
\[
  \boldsymbol{\Omega}_b = -k_2 {\bf M}_1 + k_1{\bf M}_2 .
\]
To check,
\[
    {\bf M}_1^\prime = \boldsymbol{\Omega}_b \times {\bf M}_1 = k_1 {\bf M}_2 \times {\bf M}_1 = - k_2 {\bf T} ,
\]
\[
    {\bf M}_2^\prime = \boldsymbol{\Omega}_b \times {\bf M}_2 = -k_2 {\bf M}_1 \times {\bf M}_2 = - k_2 {\bf T}.
\]

What should also be true is that
\[
        \Omega_b = \kappa {\bf B} = \sqrt{k_1^2 + k_2^2} \, {\bf B} .
\]
How can we show this? Well since
\[
   \kappa {\bf N} =  k_1 {\bf M}_1 + k_2 {\bf M}_2 = \kappa \cos \theta {\bf M}_1 + \kappa \sin\theta {\bf M_2} ,
\]
where $\theta$ is the counterclockwise angle from ${\bf M_1}$ to ${\bf N}$ ,
\begin{align*}
     {\bf T}\times \kappa {\bf N} &= {\bf T} \times  (  \kappa \cos \theta {\bf M}_1 + \kappa \sin\theta {\bf M_2}   ) \\
                                   &= k_1 {\bf M}_2 - k_2 {\bf M}_1 \\
                                   &= \boldsymbol{\Omega}_b.
\end{align*}

To see this without a computation, note that since the two frames share the vector ${\bf T}$, the motion of the Bishop Frame relative to the Frenet Frame is a rotation about ${\bf T}$ with angular velocity
\[
      \boldsymbol{\Omega}_{b/f} = -\tau {\bf T}.
\]
So
\begin{align*}
        \boldsymbol{\Omega}_{b} &= \boldsymbol{\Omega}_{b/f}  + \boldsymbol{\Omega}_f   \\
                                                &=     -\tau {\bf T} + (\tau {\bf T} + \kappa {\bf B})  \\
                                                &= \kappa {\bf B} .        
\end{align*}


\begin{exploration}
\begin{onlineOnly}
    \begin{center}
\desmosThreeD{4cjlddpmwd}{800}{600}  
\end{center}
\end{onlineOnly}

\href{https://www.desmos.com/3d/4cjlddpmwd}{163: Rotational Motion of Frenet and Bishop Frames}
\end{exploration}





\section*{Spherical Helix}
On a cylinder of radius $a$ with pitch angle $\phi$, and with 
\[
    \alpha = \frac{\cos\phi}{a},
\]
we have
%\[
%   {\bf r}(s) =  (x,y,z) = \left( a \cos \left( \frac{s\cos\phi}{a} \right) , a \sin \left( \frac{s\cos\phi}{a} \right), s \sin \phi  \right) .
%\]
\[
   {\bf r}(s) =  (x,y,z) = \left( a \cos \left( \alpha s \right) , a \sin \left( \alpha s \right), s \sin \phi  \right) .
\]
So
%\[
%     {\bf T}(s) = \left( - \cos\phi \sin \left( \frac{s\cos\phi}{a} \right)  ,\cos\phi   \cos \left( \frac{s\cos\phi}{a} \right)  , \sin \phi         \right) ,
%\]
\[
     {\bf T}(s) = \left( - \cos\phi \sin \left( \alpha s \right)  ,\cos\phi   \cos \left( \alpha s \right)  , \sin \phi         \right) ,
\]
%\[
 %        {\bf N}(s) = \left(  - \cos \left( \frac{s\cos\phi}{a} \right)  ,   -\sin \left( \frac{s\cos\phi}{a} \right) , 0    \right) ,
%\]
\[
         {\bf N}(s) = \left(  - \cos \left( \alpha s \right)  ,   -\sin \left( \alpha s \right) , 0    \right) ,
\]
and
\[
     {\bf B}(s) = \left(   \sin\phi  \sin \left( \alpha s \right) ,   -\sin\phi    \cos \left(\alpha s\right) , \cos\phi \right) .
\]
With $\theta = s\cos\phi/a$, the helix has curvature
\[
    \kappa = \frac{d\beta}{ds} = \frac{a\cos\phi \, d\theta}{a\sec\phi \, d\theta} = \frac{\cos^2\phi}{a}
\]
and torsion
\[
     \tau = \frac{d\beta^*}{ds} = \frac{a\sin\phi \, d\theta}{a\sec\phi \, d\theta} = \frac{\sin\phi\cos\phi}{a}.
\]

For the Bishop frame,
\begin{align*}
{\bf M}_1 &= \cos(-\tau s) {\bf N} + \sin(-\tau s) {\bf B}     \\
               &= \cos(\tau s) {\bf N} - \sin(\tau s) {\bf B} \\
\end{align*}
with components
\begin{align*}
     ( {\bf M}_1)_x &= - \cos(\tau s)\cos \left( \alpha s \right) - \sin(\tau s)   \sin\phi  \sin \left( \alpha s \right) \\
                           &= 
\end{align*}



\section*{Parallel Frames Look Parallel}

\begin{exploration}
\begin{onlineOnly}
    \begin{center}
\desmosThreeD{jzp8zemhz6}{800}{600}  
\end{center}
\end{onlineOnly}

\href{https://www.desmos.com/3d/jzp8zemhz6}{Cone and Parallel Vector Field}
\end{exploration}




\section*{Torus}

\begin{exploration}
\begin{onlineOnly}
    \begin{center}
\desmosThreeD{pkd5ztemml}{800}{600}  
\end{center}
\end{onlineOnly}

\href{https://www.desmos.com/3d/pkd5ztemml}{Torus Parallel Vector Field}
\end{exploration}



\section*{Curves of Constant Precession}






\end{document}