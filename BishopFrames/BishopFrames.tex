\documentclass{ximera}
\title{Bishop Frames}

\newcommand{\pskip}{\vskip 0.1 in}

\begin{document}
\begin{abstract}
This note is based on the paper \emph{There is more than one way to frame a curve}.
\end{abstract}
\maketitle


\section*{Introduction}
The key idea of the paper is captured in the first two sentences of Section 1:

\pskip

\emph{Relatively parallel fields.} We say that a normal vector field $M$ along a curve is \emph{relatively parallel} if its derivative is tangential. Such a field turns only whatever amount is necessary for it to remain normal, so it is as close to being parallel without losing normality.

\pskip

For me it helps to have a physical model or to at least imagine one. Start by bending a wire into a space curve. Then cut out a small cardboard disk and punch a hole is its center. Now \emph{slide} the disk along the wire, keeping the disk perpendicular to the curve at all times \emph{without} turning the disk about the wire.

It's the without ``turning part'' that's crucial. This is equivalent to requiring each point of the disk to move in the direction of the wire (ie. normal to the disk) at all times. Since there are infinitely many ways to slide the wire normally along the wire with rotation and only one way without, we'll start with an example of the first.


\begin{exploration}
The animation below is generated by the Frenet frame (principal normal (green) and binormal (orange). The eight points are fixed relative to this frame. The disk should appear to rotate about the cylindrical helix; the paths traced by the points are \emph{not} normal to the disk. You can take control of the slider $w$ in Line 32.

\href{https://www.desmos.com/3d/vroc5lhox8}{163: Tube around cylindrical helix 4}
\end{exploration}


\begin{exploration}
The animation below is generated by a Bishop frame frame (not shown). The eight points are fixed relative to this frame, but the principal normal (green) rotatates relative to this frame. The disk should appear to move along the cylindrical helix \emph{without} rotating; the paths traced by the points are normal to the disk. The vectors (not shown) from the disk's center to the eight points are relatively parallel to the helix. You can take control of the slider $w$ in Line 32.

\href{https://www.desmos.com/3d/fg6l58x6tl}{163: Tube around cylindrical helix 3}
\end{exploration}

\begin{exploration}
This animation starts with a tube around the cylindrical helix and a net on this tube generated by the Frenet frame. Note that the net is not orthogonal. Move the slider $p$ in Line 47 to transform the net into an orthogonal net generated by a Bishop frame.

\href{https://www.desmos.com/3d/agzievbron}{163: Tube around cylindrical helix 2}
\end{exploration}

\section*{Computing Relatively Parallel Fields}
%The image of a disk sliding along a curve without rotating

The key to computing a relatively parallel vector field along a curve is to destroy the rotation of the Frenet frame about its tangent vector $\overrightarrow{T}$ by replacing the principal normal with a normal vector relatively parallel to the curve. To do this we need to know the component of the Frenet frame's rotation about $\overrightarrow{T}$.

These components (about the unit tangent $\overrightarrow{T}$, the principal normal $\overrightarrow{N}$, and the binormal $\overrightarrow{B}$) are governed by the Frenet equations. 

By definition (using primes for differentiation with respect to arclength)
\[
       \overrightarrow{T}^\prime = \kappa \overrightarrow{N} ,
\] 
where $\kappa \geq 0$ is the curvature. From this it follows that $\overrightarrow{B}^\prime$ is also a multiple (not necessarily positive) of $\overrightarrow{B}$. 

\begin{exercise}  \label{E:Kldsfttdd}
(a) Prove algebraically that
\[
   \overrightarrow{B}^\prime = - \tau \overrightarrow{N}
\]
for some scalar $\tau$. 
\end{exercise}


\end{document}