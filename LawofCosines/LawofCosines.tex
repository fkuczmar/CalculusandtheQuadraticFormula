\documentclass{ximera}
\title{Simple Pendulums and the Law of Cosines}

\newcommand{\pskip}{\vskip 0.1 in}

\begin{document}
\begin{abstract}
We use calculus to prove the law of cosines and show how this is related to revolving pendulums.
\end{abstract}
\maketitle

There is a close relationship between the law of cosines and the motion of a simple pendulum that revolves about its center.

We'll start by using calculus to prove the law of cosines. First some preliminaries.

The differential arclength along the curve $y=f(x)$ is
\begin{align*}
   ds   &= \sqrt{dx^2 + dy^2} \\
         & = \sqrt{1+\left( dy/dx  \right)^2} \, dx \\
         &= \sec\alpha^* \, dx \\
         &= \csc \alpha \, dx
\end{align*}
where $\phi$, $-\pi/2 < \alpha^* <\pi/2$ is the angle of incination (ie. the acute angle the curve makes with the horizontal) and $\alpha = \pi/2 - \alpha^*$ is the angle the curve makes with the vertical.

The differential arclength along the polar curve $\rho = f(\phi)$ is almost identical. Here
\begin{align*}
   |ds|   &= \sqrt{(\rho \, d\phi)^2 + d\rho^2} \\
         & = \sqrt{1 + \left( \frac{\rho\, d\phi}{d\rho}\right)^2} \, d\rho \\
         %&= \sec\alpha^* \, dx \\
         &= \csc \alpha \, d\rho ,
\end{align*}
where $\alpha$, $0<\alpha \leq \pi/2$ is the angle the curve makes with the polar radius.

For the law of cosines in $\Delta AOP$, we'll let $\phi$ be the measure of the \emph{exterior} angle at $O$, $a=OA$, $R = OP$, $\rho = AP$, and $\alpha = \angle OAP$. Then we wish to show
\[
        \rho^2 = a^2 + R^2 + 2aR\cos\phi .
\]
We'll do this by holding $R$ and $a$ constant and showing
\[
   \frac{d}{d\phi} \left(   \rho^2 - 2aR\cos\phi  \right) = 0 .
\]

Computing the derivative,
\begin{align*}
  \frac{d}{d\phi} \left(   \rho^2 - 2aR\cos\phi  \right)  &= 2\rho \frac{d\rho}{d\phi}+ 2aR\sin\phi  \\
                              &=2a \left(  \frac{\rho}{a} \cdot \frac{d\rho}{d\phi}   + R\sin\phi  \right) \\
                              &= 2a \left(  \frac{\sin\phi}{\sin\alpha} \cdot \frac{d\rho}{d\phi}  + R\sin\phi   \right)\\
\end{align*}





\begin{exploration}
Zoom in on the picture below. What do you see?
\begin{onlineOnly}
    \begin{center}
\desmos{awoec8d6pq}{800}{600}  
\end{center}
\end{onlineOnly}

\href{https://www.desmos.com/calculator/awoec8d6pq}{163: Rock Art}
\end{exploration}


\begin{exploration}
Click and rotate the image so the orange vector points directly toward you. What happens? Experiment with the sliders, zooming out as need be.
\begin{onlineOnly}
    \begin{center}
\desmosThreeD{9g2timus2m}{800}{600}  
\end{center}
\end{onlineOnly}

\href{https://www.desmos.com/3d/9g2timus2m}{163: Rock Art 1}
\end{exploration}


\end{document}