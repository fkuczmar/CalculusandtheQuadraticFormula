\documentclass{ximera}
\title{The Limits to Growth}

\newcommand{\pskip}{\vskip 0.1 in}


\begin{document}
\begin{abstract}
Depletion of the world's resources.
\end{abstract}
\maketitle

%\begin{image}  
%\includegraphics[width=\textwidth]{Graph1.png}  
%\end{image}

\emph{The Project on the Predicament of Mankind} was initiated in 1968 by a group of 30 individuals having a wide range of expertise with the intent of examining the complex problems troubling all nations at that time and that still persist today. Within a year thay had created a model of the five basic factors they identified as limiting growth  - population, agricultural production, natural resources, industrial production, and polution. While perhaps simple by today's standards, the model nevertheless provided valuable insights into how these areas interact with each other to limit growth. The findings are summarized in the 1972 classic \emph{The Limits to Growth}. 


\pskip

Some of the most striking observations occur in Chapter 2, \emph{The Limits to Exponential Growth}, when the authors analyze the effects of population growth on the demand for arable land (see the figure below), writing 

\pskip

\emph{The figure also illustrates some very important facts about exponential growth in a limited space. First, it shows how one can move within a very few years from a situation of great abundance to one of great scarcity.}  

\pskip

And what might be my favorite sentence of the book, 

\pskip

\emph{A second lesson to be learned from the figure is that precise numerical assumptions about the limits of the earth are unimportant when viewed against the inexorable progress of exponential growth.}

\pskip

\pdfOnly{
Access Desmos interactives through the online version of this text at
 
\href{https://www.desmos.com/calculator/ovjygr01yy}.
}
 
\begin{onlineOnly}
    \begin{center}
\desmos{ovjygr01yy}{900}{600}
\end{center}
\end{onlineOnly}


The focus of this note, however, is about their analysis of the depletion of the world's chromium reserves as illustrated in the figure below. %This model might be turned into a meaningful project for a Math 152 class.


\pdfOnly{
Access Desmos interactives through the online version of this text at
 
\href{https://www.desmos.com/calculator/9kqmbczco7}.
}
 
\begin{onlineOnly}
    \begin{center}
\desmos{9kqmbczco7}{900}{600}
\end{center}
\end{onlineOnly}

\pskip \pskip

Chromium is just one of 19 non-renewable natural resources that they analyze. For each, the authors provide the following information.

\begin{itemize}
\item{The known global reserves in 1970; for chromium $7.75\times 10^8$ tons}

\item{The static index. This is the number of years it the global reserves would last at the current global consumption rate.}

\item{The projected rate of growth in the global consumption rate. For chromium, the nominal rate is $2.6\%$/yr, compounded continuously.}

\end{itemize}

For each resource, the authors compute the \emph{exponential index}. This is the number of years the known current reserves would last with consumption growing exponentially at the average rate of growth. They also compute the exponential index using five times the known reserves. 

\begin{question}   \label{Q34fratr5}
Let's work in general (like the authors) and let $r$ be the nominal annual growth rate in the global consumption rate and let $s$ be the static index. Our goal is to express  the exponential index in terms of $r$ and $s$.  

\pskip 

(a) We'll start by computing the usage rate function (see the graph above)
\[
     u = f(t) , 0\leq t \leq 2230,
\]
that expresses the usage rate (measured in $10^8$ tons/year) in terms of the number of years since 1970. We'll do this in general. letting $R$ be the 1970 global reserves (measured in units of $10^8$ tons). Try this on your own first and input your expression in the box below. Otherwise, click the Hint button for some help.
\[
   u = f(t) = \left( \answer{\frac{R}{rs}} \right) (e^{\answer{rt}} - \answer{1}) , 0\leq t \leq 2230 .
\]




\end{question}





\end{document}
