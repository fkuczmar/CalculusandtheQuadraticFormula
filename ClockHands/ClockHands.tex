\documentclass{ximera}
\title{Clock Hands}

\newcommand{\pskip}{\vskip 0.1 in}

\begin{document}
\begin{abstract}
A twist on a familiar problem.
\end{abstract}
\maketitle

\section*{Introduction}
A problem in Stewart asks to find the rate at which the distance between the tips of the minute and hour hands of a clock is changing at a specific time. But a more interesting question is to ask when this distance is increasing at the fastest rate. 


\begin{exploration}
(a) Experiment with the animation below, perhaps changing the lengths of the hands. Any conjectures?

(b) The distance function looks sinusoidal, but opening the folder in Line 26 to see the graph of the derivative dispels that notion. 

\begin{onlineOnly}
    \begin{center}
\desmos{zz3qff7ocv}{450}{600}  %qvk0mzy26u
\end{center}
\end{onlineOnly}

\href{https://www.desmos.com/calculator/zz3qff7ocv}{151: Hands of a Clock}


\end{exploration}

\section*{A Computational Solution}
%It simplifies things if we ignore time and work in the reference frame of one of the hands

We'll let $a$ and $b$ denote the respective lengths of the hour and minute hands, $\theta$ the angle (in radians) between the hands, and $c$ the distance between their tips. Ignoring time for the moment, our first step is to maximize the derivative,
\[
        \frac{dc}{d\theta} = \frac{ab\sin \theta}{c} = \frac{ab \sin \theta}{\sqrt{a^2+b^2-2ab\cos\theta}}.
\]
So we differentiate again and do some algebra to get
\[
       \frac{d^2c}{d\theta^2} = \frac{ab}{c^3} \left( c^2\cos\theta - ab\sin^2\theta  \right).
\]
Then setting the second derivative equal to zero and substituting 
\[
      c^2 = a^2+b^2-2ab\cos\theta
\]
leads to the equation (symmetric in $a$ and $b$)
\[
     ab \cos^2\theta - (a^2 + b^2)\cos\theta + ab = 0
\]
or
\[
              (a \cos \theta - b) (b \cos \theta - a) = 0 .
\]
Assuming the minute hand is longer than the hour hand ($b>a$) and $0\leq \theta \leq 2pi$, we conclude that the distance is increasing at the fastest rate when $\theta = \arccos(a/b)$ and decreasing at the fastest rate when $\theta  = 2\pi - \arccos(a/b)$.

\section*{A More Geometrical Solution}
We could have saved ourselves the trouble of taking the second derivative by interpreting the rate of change
\[
    \frac{dc}{d\theta} = \frac{ab\sin \theta}{c} 
\]
in the distance geometrically. To do this, note that since $ab\sin\theta = 2K$ is twice the area of $\Delta ABC$,
\[
      \frac{dc}{d\theta} = \frac{2K}{c}  = h,
\]
where $h$ is the distance from from $C$ to $\overrightarrow{AB}$.
\end{document}
