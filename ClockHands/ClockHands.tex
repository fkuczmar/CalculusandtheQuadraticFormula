\documentclass{ximera}
\title{Clock Hands}

\newcommand{\pskip}{\vskip 0.1 in}

\begin{document}
\begin{abstract}
A twist on a familiar problem.
\end{abstract}
\maketitle

\section*{Introduction}
We're all familiar with the problem about finding the rate at which the distance between the tips of the minute and hour hands of a clock is changing at a certain time. But a more interesting question, for both Math 151 and 163, is to ask when this distance is increasing at the fastest rate. 


\begin{exploration}
Experiment with the animation below, perhaps changing the lengths of the hands. Any conjectures?

\begin{onlineOnly}
    \begin{center}
\desmos{qvk0mzy26u}{450}{600} 
\end{center}
\end{onlineOnly}

\href{https://www.desmos.com/calculator/qvk0mzy26u}{151: Hands of a Clock}


\end{exploration}

\section*{A Computational Solution}
%It simplifies things if we ignore time and work in the reference frame of one of the hands

We'll let $a$ and $b$ denote the respective lengths of the hour and minute hands, $\theta$ the angle (in radians) between the hands, and $c$ the distance between their tips. Then by the law of cosines,
\[
        \frac{dc}{dt} = \frac{ab\cos \theta}{c} = \frac{ab cos\theta}{\sqrt{a^2+b^2-2ab\cos\theta}} ,
\]
where $t$ is the time, say measured in hours.

\end{document}
