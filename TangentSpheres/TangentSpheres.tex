\documentclass{ximera}
\title{Mutually Tangent Spheres}

\newcommand{\pskip}{\vskip 0.1 in}

\begin{document}
\begin{abstract}
Two spheres on a table top are tangent to each other. We describe the center set of all spheres on the table that are tangent to these two. 
\end{abstract}
\maketitle

Get two spheres, say an orange and a grapefruit, and place them on a table so that they are tangent to each other. Now place a third sphere, a tangerine perhaps, on the table and tangent to the other two. Find a fourth sphere if you can and place it on the table tangent to the orange and grapefruit. Keep going, and try to imagine the set of all spheres on the table that are tangent to the orginal two (do this in your mind so that you can place all these spheres on the table at once). 

What can we say about the set of centers, what we'll call the \emph{center set} of all these variable spheres? Is the set finite? Are there enough centers to form a curve? Or maybe a surface?

Some other questions to ponder. Is there a smallest sphere tangent to the original two? A largest? Is the center set bounded or not?

Can you guess what the center set might look like? Or what about the \emph{tangent-set}, those points where the variable spheres are tangent to the table top?

%We'll put these questions aside for now and get a feel for them by thinking about an analogous problem in two dimensions.

We'll set these questions aside for now and first think about a similar question in the plane.

\section{Circles on a Line}
\begin{exploration}
Let's replace the table with a line, say the $x$-axis, and the two original spheres with just one circle ${\cal C}_1$ tangent to the $x$-axis. What can we say about the center set of all circles tangent to the $x$-axis and ${\cal C}_1$?

One point before we get started. We really require that the variable circle be externally tangent to ${\cal C}_1$ (can you see why) and we assume this throughout.

To explore this question, drag the centers of the circles below to make them tangent to the unit circle $x^2 + (y-1)^2=1$.

Do  you have a guess about what the center set might be?

\pdfOnly{
Access Geogebra interactives through the online version of this text at
 
\href{https://www.geogebra.org/classic/egzphw3q}.
}
 
\begin{onlineOnly}
    \begin{center}
\geogebra{egzphw3q}{900}{600}
\end{center}
\end{onlineOnly}
\end{exploration}



\begin{exploration}
To check your guess, we'll replace the unit circle with a circle ${\cal C}_1$ with radius $r_1$ and centered at $A(0,r_1)$. What is the center-set of the circles  tangent to the $x$-axis and ${\cal C}_1$?

 %let's first find a geometric description of the center set for circles tangent to both the $x$-axis and the circle ${\cal C}_1$ with radius $r_1$ centered at $A(0,r_1)$.


\begin{question}  \label{Q243:Spheres}
Use the picture below to describe a necessary and sufficient condition for the point $P$ to be the center of a circle tangent to both the $x$-axis and the circle ${\cal C}_1$. Choose all that apply.

\begin{selectAll}  
    \choice{The point $P$ is equidistant from $A$ and the $x$-axis.}  
    \choice[correct]{The distance from $P$ to $Q$ is equal to the sum of the radii of the two circles.}  
    \choice{The distance from $P$ to $A$ is $r_1$ units less than the distance from $A$ to the $x$-axis.}  
    \choice[correct]{The distance from $P$ to $A$ is $r_1$ units greater than the distance from $A$ to the $x$-axis.}  
     \choice[correct]{$P$ is equidistant from the point $A$ and the line $y=-r_1$.}
  \end{selectAll}  

\end{question}


\pdfOnly{
Access Desmos interactives through the online version of this text at
 
\href{https://www.desmos.com/calculator/nnryp5eamp}.
}
 
\begin{onlineOnly}
    \begin{center}
\desmos{nnryp5eamp}{900}{600}
\end{center}
\end{onlineOnly}
\end{exploration}


\begin{exploration}

Our next step is to turn one of the geoemtric descriptions above (take your pick) into an equation for the center set. Do this, do some algebra to simplify the equation, and write your simplified equation of the center set below.
\begin{question} \label{Q341:Spheres}
\[
    y = \answer{\frac{x^2}{4r_1}}
\]
\end{question}


Next use your equation of the center set to find equations of the two circles with radius $r>0$  tangent to both the $x$-axis and ${\cal C}_1$. Enter these below.

\begin{question} \label{Q4541:Spheres}
An equation of the circle centered in the first quadrant is
\[
     \answer{(x-2\sqrt{rr_1})^2 + (y-r)^2 = r^2} .
\]
An equation of the circle centered in the second quadrant is
\[
     \answer{(x+2\sqrt{rr_1})^2 + (y-r)^2 = r^2} .
\]
\end{question}

Finally, follow the directions in the desmos activity below. Be sure to drag the siders to check that your variable circles stay tangent to ${\cal C}_1$ and the $x$-axis.


\pdfOnly{
Access Desmos interactives through the online version of this text at
 
\href{https://www.desmos.com/calculator/nixr157hq7}.
}
 
\begin{onlineOnly}
    \begin{center}
\desmos{nixr157hq7}{900}{600}
\end{center}
\end{onlineOnly}

\pskip

\begin{question} \label{Q441:Spheres}
What happens to the center set in the above demonstation when the radius of the fixed circle is zero (ie. when $r_1=0$)? Is this correct? If not, what should the center set be in this case?
\begin{hint}
The center set looks like it vanishes when $r_1=0$. But this is not correct. Think about what it would mean for a circle to be tangent to the degenerate circle $(0,0)$ with radius $r_1=0$. Use this interpretation to describe the center set correctly.
\end{hint}

\end{question}
\end{exploration}


\begin{exploration}   \label{Ex2434634:Spheres}
Now let's fix one of the variable circles from the last exercise, the one centered in first quadrant, and let it have radius $r_2$ instead of $r$. Then we have two fixed circles
\[
   {\cal C}_1: x^2 + y^2 = r_1^2
\]
and
\[
  {\cal C}_2:  (x-2\sqrt{r_1r_2})^2 + (y-r_2)^2 = r_2^2
\]
tangent to the $x$-axis and to each other. Our problem now is to find equations of all circles tangent to the $x$-axis, ${\cal C}_1$ and ${\cal C}_2$. How many of these circles do you think there are?


%\begin{question}   \label{Q5693:Spheres}
Our first step is to describe the center set for the circle ${\cal C}_2$. Write an equation for this set below. You might be able to guess an equation based our previous work. If not, describe the set geometrically and then translate this description into an equation. 
\[
    \answer{y=\frac{(x-2\sqrt{r_1r_2})^2}{4r_2}}
\]
%\end{question}

Now the center of a third circle tangent to the first two and the $x$-axis must lie on both center sets. Use this information to find the coordinates of the centers of the two possible circles. Input these here (the center with the smaller $x$-coordinate first) and in the desmos activity below.
\[
   \left( \answer{\frac{2r_1\sqrt{r_2}}{\sqrt{r_1} -\sqrt{r_2}}} , \answer{\frac{r_1 r_2}{(\sqrt{r_1} -\sqrt{r_2})^2}}  \right) 
\]

\[
   \left( \answer{\frac{2r_1\sqrt{r_2}}{\sqrt{r_1} +\sqrt{r_2}}} , \answer{\frac{r_1 r_2}{(\sqrt{r_1} +\sqrt{r_2})^2}}  \right) 
\]


\pdfOnly{
Access Desmos interactives through the online version of this text at
 
\href{https://www.desmos.com/calculator/j96fr7drbv}.
}
 
\begin{onlineOnly}
    \begin{center}
\desmos{grf1kxmlem}{900}{600}
\end{center}
\end{onlineOnly}
\end{exploration}

Just to recap:

\begin{itemize}
\item{The center set of circles tangent to both the $x$-axis and the circle
\[
     {\cal C}_1: x^2 + (y-r_1)^2 = r_1^2
\] 
is the parabola
\[
    {\cal P}_1: 4r_1 y = x^2 .
\]
}

\item{The center set of circles tangent to both the $x$-axis and the circle
\[
     {\cal C}_2: (x-2\sqrt{r_1r_2})^2 + (y-r_2)^2 = r_2^2
\] 
is the parabola
\[
    {\cal P}_2: 4r_2 y = (x-2\sqrt{r_1r_2})^2 .
\]
}

\item{If $r_1\neq r_2$, there are two circles tangent to ${\cal C}_1$, ${\cal C}_2$, and the x-axis, with centers
\[
      \left( \frac{2r_1\sqrt{r_2}}{\sqrt{r_1} -\sqrt{r_2}} , \frac{r_1 r_2}{(\sqrt{r_1} -\sqrt{r_2})^2}  \right) 
\]
and
\[
      \left( \frac{2r_1\sqrt{r_2}}{\sqrt{r_1} +\sqrt{r_2}} , \frac{r_1 r_2}{(\sqrt{r_1} +\sqrt{r_2})^2}  \right) .
\]
The $y$-coordinates of these centers give the respective radii.
}
\end{itemize}


\begin{question} \label{Q43ttgt:Spheres}
What happens if $r_1=r_2$?
\end{question}

One final point. There is a relationship among the curvatures $\kappa_i$ of four mutually tangent circles in the plane, namely that
\[
    \left( \sum_{i=0}^3 \kappa_i \right)^2 = 2 \sum_{i=0}^3 \kappa_i^2 .     \label{Eq:Q1TangentSpheres}
\]

%Or, as Frederick Soddy wrote in the poem \it{The Kiss Precise}

%\pskip

%The sum of the squares of all four bends 

%Is half the square of their sum

%\pskip

As a special case, we can regard the $x$-axis as our fourth circle with curvature $\kappa_0 = 0$. We can verify this curvature relationship for each of our two families of four mutually tangent circles by solving the quadratic equation above for $\kappa_3$. The result is that
\begin{align*}
   \kappa_3   &= \kappa_1 + \kappa_2 \pm 2\sqrt{\kappa_1\kappa_2}  \\
                   &= \frac{r_1 r_2}{(\sqrt{r_1} \pm\sqrt{r_2})^2} ,
\end{align*}
as we just saw.

For a lot more on this see

\href{https://en.wikipedia.org/wiki/Descartes%27_theorem}{Decartes' Theorem}


\section{Spheres on a Table}
We now return to our orginal question about mutually tangent spheres on a table. Given two such spheres, say of radii $r_1$ and $r_2$, we wish to describe the center set of all spheres on the table tangent to these two. 

To get an idea of the dimension of the center set, it helps to think about degrees of freedom. We have four degrees of freedom in choosing a sphere in space - three for the coordinates of the center and one for the radius. Now our third sphere must meet three conditions - that it be tangent to the first two spheres and to the table top. This leaves one degree of freedom and we should expect the center set to be a curve in $\mathbb{R}^3$. What curve?



\begin{question}    \label{Q23423:Spheres}
  
Taking much the same approach as in two-dimensions, we'll start by finding the center set of the sphere %with radius $r_1$
\[
      {\cal S}_1:    x^2 + y^2 + (z-r_1)^2 = r_1^2 .
\]
With two degrees of freedom left, we would expect the center-set of spheres tangent to ${\cal S}_1$ and the $xy$-plane to be a surface. How would you describe this set geometrically?
 
  \begin{selectAll}  
    \choice{The set of points equidistant from the $xy$-plane and the point $(0,0,r_1)$.}  
    \choice[correct]{The set of points equidistant from the plane $z=-r_1$ and the point $(0,0,r_1)$.}  
    \choice{The set of points equidistant from the plane $z=r_1$ and the point $(0,0,r_1)$.}  
    \choice{The set of points equidistant from the $z$-axis and the point $(0,0,r_1)$.}  
  \end{selectAll}  

\pskip

Translate your geometric description into an equation of the center-set (a paraboloid of revolution) and enter it below.
\[
   {\cal P}_1:  \answer{4 r_1z  = x^2 + y^2}
\]
\end{question}


\begin{question}  \label{Q365434:Spheres}
Let's add a second sphere of radius $r_2$, tangent to ${\cal S}_1$ and the $xy$-plane. To simplify matters, we'll make this sphere tangent to $xy$-plane at a point on the positive $x$-axis. Its equation is much like that of the second circle ${\cal C}_2$ in the plane. Enter an equation of this second sphere below.
\[
   {\cal S}_2: \answer{(x-2\sqrt{r_1r_2})^2 + y^2 + (z-r_2)^2 = r_2^2}
\]

Next write an equation of the center-set for the spheres tangent to ${\cal S}_2$ and the $xy$-plane.
\[
   {\cal P}_2:  \answer{4r_2z = (x-2\sqrt{r_1r_2})^2 + y^2}
\]

\end{question}



We know now that the center-set of spheres tangent to the $xy$-plane and the spheres ${\cal S}_1$, ${\cal S}_2$ is the set of points on both the paraboloids of revolution
\[
   {\cal P}_1:  4 r_1z  = x^2 + y^2
\]
and 
\[
     {\cal P}_2:  4r_2z = (x-2\sqrt{r_1r_2})^2 + y^2 .
\]

This makes the center-set the curve in which these paraboloids intersect. It seemed to me that the paraboloids, being unbounded, would intersect in an unbounded curve. But you might pause here and try to visualize this curve for yourself.


\begin{exploration}  \label{Exsdsdfsdf:Spheres}
Experiment with the sliders below and watch how the center-set changes. 

(a) Make a conjecture about the center-set.

(b) Make a conjecture about the tangent-set (the set of points where the spheres are tangent to the $xy$-plane).

\pdfOnly{
Access Geogebra interactives through the online version of this text at
 
\href{https://www.geogebra.org/classic/m4zhhshd}.
}
 
\begin{onlineOnly}
    \begin{center}
\geogebra{m4zhhshd}{900}{600}
\end{center}
\end{onlineOnly}


\end{exploration}


\begin{question} \label{Q5r43rd:Spheres}

Our next step is to give precise descriptions of the center and tangent sets. It would be nice to do this geometrically, but I don't know how. So instead, we'll continue with our algebraic approach.

(a) Using the equations of the two paraboloids above, eliminate the variable $y$ to get a relation between $x$ and $z$. Write this relation below in a simplfied form that you can recognize and that works even if $r_1 = r_2$.

\[
    \answer{\sqrt{r_1 r_2}x + (r_2 - r_1) z = r_1 r_2}
\]

(b) Now find a relationship between the variables $x$ and $y$ by using your equation from part (a) and the equation of the paraboloid ${\cal P}_1$. At this point it's probably best not to do too much algebra. Instead write the relation in the form 
\[
   ax^2 + bx + cy^2 = e 
\]
and enter it below.
\[
    \answer{(r_2 - r_1)x^2 + 4 r_1\sqrt{r_1r_2}x + (r_2-r_1)y^2 =  4r_1^2 r_2}
\]

(c) What can you conclude from the two equations above if $r_1\neq r_2$? Check all that apply.
 \begin{selectAll}  
        \choice[correct]{The center set lies on a plane parallel to the $y$-axis.}  
    \choice[correct]{The center set lies on circular cylinder with generating lines parallel to the $z$-axis.} 
    \choice[correct]{The tangent set is a circle.}    
    \choice[correct]{The center set is an ellipse.}  
  \end{selectAll}  

(d) What can you conclude from the two equations above if $r_1= r_2$? Check all that apply.
 \begin{selectAll}  
    \choice[correct]{The tangent set is a line.}  
    \choice[correct]{The tangent set lies on the plane $x = \sqrt{r_1r_2}$.}  
    \choice{The center set lies on circular cylinder symmetric about a line parallel to the $z$-axis.} 
    \choice{The center set is an ellipse.}  
     \choice[correct]{The center set is a parabola.} 
  \end{selectAll}



\pdfOnly{
Access Geogebra interactives through the online version of this text at
 
\href{https://www.geogebra.org/classic/t9rxbmmt}.
}
 
\begin{onlineOnly}
    \begin{center}
\geogebra{t9rxbmmt}{900}{600}
\end{center}
\end{onlineOnly}


\end{question}


\pskip
 
For a Desmos version of a similar animation:

\href{https://www.desmos.com/3d/7fb88aea41}{Mutually Tangent Spheres}


To summarize what we know so far:

\begin{itemize}

\item{The center set is the curve where the plane
\[
   \Pi:  \sqrt{r_1 r_2}x + (r_2 - r_1) z = r_1 r_2
\]
intersects the paraboloid
\[
  {\cal P}_1:  4r_1z = x^2 + y^2 .
\]
It is an ellipse if $r_1\neq r_2$ and a parabola otherwise.
}

\item{The center set is also the curve where the above plane intersects the surface
\[
       (r_2 - r_1)x^2 + 4 r_1\sqrt{r_1r_2}x + (r_2-r_1)y^2 =  4r_1^2 r_2 .
\]
The surface is a circular cylinder if $r_1\neq r_2$. But if $r_1=r_2$ the surface is the plane $x=\sqrt{r_1r_2}$. 
}

\end{itemize}

To animate the variable third sphere, we need only parameterize the center set and the family of variable tangent spheres. We turn to this next.

One approach would be to use the cosine and sine functions to paramaterize the tangent set when $r_1\neq r_2$ and then lift this parameterization to the ellipse. But it takes some work to find the center and radius of the tangent set and this parameterization fails when $r_1=r_2$. For the record, when $r_1\neq r_2$, the tangent set is a circle with radius
\[
     \frac{2r_1r_2}{|r_1 - r_2|}
\] 
centered at
\[
   \left( \frac{2r_1 \sqrt{r_1r_2}}{r_1-r_2} , 0   \right) .
\]

We'll take a simpler approach and parameterize the center set by the radius $r$ of the variable sphere. This has the advantage of working when $r_1=r_2$, but the disadvantage of needing to split the parameterization into two parts (since there are usually two variable spheres, symmetric about the $xz$-plane, with a given radius).



\end{document}

