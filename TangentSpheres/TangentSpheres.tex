\documentclass{ximera}
\title{Mutually Tangent Spheres}

\newcommand{\pskip}{\vskip 0.1 in}

\begin{document}
\begin{abstract}
Two spheres on a table top are tangent to each other. We describe the center set of all spheres on the table that are tangent to these two. 
\end{abstract}
\maketitle

Get two spheres, say an orange and a grapefruit, and place them on a table so that they are tangent to each other. Now place a third sphere, a tangerine perhaps, on the table and tangent to the other two. Find a fourth sphere if you can and place it on the table tangent to the orange and grapefruit. Keep going, and try to imagine the set of all spheres on the table that are tangent to the orginal two (do this in your mind so that you can place all these spheres on the table at once). 

What can we say about the set of centers, what we'll call the \emph{center set} of all these variable spheres? Is the set finite? Are there enough centers to form a curve? Or maybe a surface?

Some other questions to ponder. Is there a smallest sphere tangent to the original two? A largest? Is the center set bounded or not?

Can you guess what the center set might look like? Or what about the set of \emph{tangency points}, where the variable spheres are tangent to the table top?

We'll put these questions aside for now and get a feel for these questions by thinking about an analogous problem in two dimensions.

\section{Circles on a Line}
\begin{exploration}
Let's replace the table with a line, say the $x$-axis, and the two original spheres with just one circle ${\cal C}_1$ tangent to the $x$-axis. What can we say about the center set of all circles tangent to the $x$-axis and ${\cal C}_1$?

One little point before we get started. We really require that the variable circle be externally tangent to ${\cal C}_1$ (can you see why) and we assume this troughout.

To explore this question, drag either the centers or the points of tangency of the circles below to make them tangent to the circle $x^2 + (y-1)^2=1$.

Do  you have a guess about what the center set might be?

\pdfOnly{
Access Geogebra interactives through the online version of this text at
 
\href{https://www.geogebra.org/classic/egzphw3q}.
}
 
\begin{onlineOnly}
    \begin{center}
\geogebra{egzphw3q}{900}{600}
\end{center}
\end{onlineOnly}
\end{exploration}



\begin{exploration}
To verify your guess, let's first find a geometric description of the center set for circles tangent to both the $x$-axis and the circle ${\cal C}_1$ with radius $r_1$ centered at $A(0,r_1)$.


\begin{question}  \label{Q243:Spheres}
Use the picture below to describe a necessary and sufficient condition for the point $P$ to be the center of a circle tangent to both the $x$-axis and the circle ${\cal C}_1$. Choose all that apply.

\begin{selectAll}  
    \choice{The point $P$ is equidistant from $A$ and the $x$-axis.}  
    \choice[correct]{The distance from $P$ to $Q$ is equal to the sum of the radii of the two circles.}  
    \choice{The distance from $P$ to $A$ is $r_1$ units less than the distance from $A$ to the $x$-axis.}  
    \choice[correct]{The distance from $P$ to $A$ is $r_1$ units greater than the distance from $A$ to the $x$-axis.}  
     \choice[correct]{$P$ is equidistant from the point $A$ and the line $y=-r_1$.}
  \end{selectAll}  

\end{question}


\pdfOnly{
Access Desmos interactives through the online version of this text at
 
\href{https://www.desmos.com/calculator/nnryp5eamp}.
}
 
\begin{onlineOnly}
    \begin{center}
\desmos{nnryp5eamp}{900}{600}
\end{center}
\end{onlineOnly}
\end{exploration}


\begin{exploration}

Our next step is to turn one of the geoemtric descriptions above (take your pick) into an equation for the center set. Do this, do some algebra to simplify the equation, and write your simplified equation of the center set below.
\begin{question} \label{Q341:Spheres}
\[
    y = \answer{\frac{x^2}{4r_1}}
\]
\end{question}


Next use your equation of the center set to find equations of all circle with radius $r$  tangent to both the $x$-axis and ${\cal C}_1$. Enter these below.

\begin{question} \label{Q4541:Spheres}
An equation of the circle centered in the first quadrant is
\[
     \answer{(x-2\sqrt{rr_1})^2 + (y-r)^2 = r^2} .
\]
An equation of the circle centered in the second quadrant is
\[
     \answer{(x+2\sqrt{rr_1})^2 + (y-r)^2 = r^2} .
\]
\end{question}

Finally, follow the directions in the desmos activity below. Be sure to drag the siders to check that your variable circles behave as they should.


\pdfOnly{
Access Desmos interactives through the online version of this text at
 
\href{https://www.desmos.com/calculator/nixr157hq7}.
}
 
\begin{onlineOnly}
    \begin{center}
\desmos{nixr157hq7}{900}{600}
\end{center}
\end{onlineOnly}

\pskip

\begin{question} \label{Q441:Spheres}
What happens to the center set in the above demonstation when the radius of the fixed circle is zero (ie. when $r_1=0$)? Is this correct? If not, what should the center set be in this case?
\begin{hint}
The center set looks like it vanishes when $r_1=0$. But this is not correct. Think about what it would mean for a circle to be tangent to the degenerate circle $(0,0)$ with radius $r_1=0$. Use this interpretation to describe the center set correctly.
\end{hint}

\end{question}
\end{exploration}


\begin{exploration}
Now let's fix the circle from the last exercise, the one centered in first quadrant, and let it have radius $r_2$ instead of $r$. Then we have two fixed circles
\[
   {\cal C}_1: x^2 + y^2 = r_1^2
\]
and
\[
  {\cal C}_2:  (x-2\sqrt{r_1r_2})^2 + (y-r_2)^2 = r_2^2
\]
tangent to the $x$-axis and to each other. Our problem now is to find equations of all circles tangent to the $x$-axis and these two fixed circles. How many of these circle do you think there are?


\begin{question}   \label{Q5693:Spheres}
Our first step is to describe the center set for the circle ${\cal C}_2$. Write an equation for this parabola below. You might be able to guess an equation based our previous work. If not, describe the set geometrically and then translate this description into an equation. 
\[
    \answer{y=\frac{(x-2\sqrt{r_1r_2})^2}{4r_2}}
\]
\end{question}

Now the center of a third circle tangent to the first two and the $x$-axis must lie on both center sets. Use this information to find the coordinates of the centers of the two possible third circles. Input these here (the center with the smaller $x$-coordinate first) and in the desmos activity below.
\[
   \left( \answer{\frac{2r_1\sqrt{r_2}}{\sqrt{r_1} -\sqrt{r_2}}} , \answer{\frac{r_1 r_2}{(\sqrt{r_1} -\sqrt{r_2})^2}}  \right) 
\]

\[
   \left( \answer{\frac{2r_1\sqrt{r_2}}{\sqrt{r_1} +\sqrt{r_2}}} , \answer{\frac{r_1 r_2}{(\sqrt{r_1} +\sqrt{r_2})^2}}  \right) 
\]


\pdfOnly{
Access Desmos interactives through the online version of this text at
 
\href{https://www.desmos.com/calculator/j96fr7drbv}.
}
 
\begin{onlineOnly}
    \begin{center}
\desmos{j96fr7drbv}{900}{600}
\end{center}
\end{onlineOnly}
\end{exploration}


There is a relationship among the curvatures of four mutually tangent circles in the plane, that
\[
    \left( \sum_{i=0}^3 \kappa_i \right)^2 = 2 \sum_{i=0}^3 \kappa_i^2 .     \label{Eq:Q1TangentSpheres}
\]

Or, as Frederick Soddy wrote in the poem \it{The Kiss Precise}

\pskip

The sum of the squares of all four bends 

Is half the square of their sum

\pskip

We can regard the $x$-axis as our fourth circle with curvature $\kappa_0 = 0$. It looks difficult to verify this relationship even for our two families of mutually tangent circles. The better approach would be to solve the quadratic equation above for $\kappa_3$ to get
\begin{align*}
   \kappa_3   &= \kappa_1 + \kappa_2 \pm 2\sqrt{\kappa_1\kappa_2}  \\
                   &= \frac{r_1 r_2}{(\sqrt{r_1} \pm\sqrt{r_2})^2} ,
\end{align*}
as we just saw above.

For a lot more on this see

\href{https://en.wikipedia.org/wiki/Descartes%27_theorem}{Decartes' Theorem}


\section{Spheres on a Table}
We now return to our orginal question about mutually tangent spheres on a table. Given two such spheres, say of radii $r_1$ and $r_2$, we wish to describe the center set of all spheres on the table tangent to these two. 

To get an idea of the dimension of the center set, it helps to think about degrees of freedom. We have four degrees of freedom in choosing a sphere in space - three for the coordinates of the center and one for the radius. Now our third sphere must meet three conditions - that it be tangent to the first two spheres and the table top. This leaves one degree of freedom and we should expect the center set to be a curve in $\mathbb{R}^3$. What curve?



\begin{question}    \label{Q23423:Spheres}
  
Taking much the same approach as in two-dimensions, we'll start by finding the center set of the sphere %with radius $r_1$
\[
      {\cal S_1}:    x^2 + y^2 + (z-r_1)^2 = r_1^2 .
\]
With two degrees of freedom left, we would expect the center-set of spheres tangent to ${\cal S}_1$ and the $xy$-plane to be a surface. How would you describe this set geometrically?
 
  \begin{selectAll}  
    \choice{The set of points equidistant from the $xy$-plane and the point $(0,0,r_1)$.}  
    \choice[correct]{The set of points equidistant from the plane $z=-r_1$ and the point $(0,0,r_1)$.}  
    \choice{The set of points equidistant from the plane $z=r_1$ and the point $(0,0,r_1)$.}  
    \choice{The set of points equidistant from the $z$-axis and the point $(0,0,r_1)$.}  
  \end{selectAll}  

\pskip

Now translate your geometric description into an equation of the center set and enter it below.
\[
    \answer{4 r_1z  = x^2 + y^2}
\]
\end{question}





\pdfOnly{
Access Geogebra interactives through the online version of this text at
 
\href{https://www.geogebra.org/classic/m4zhhshd}.
}
 
\begin{onlineOnly}
    \begin{center}
\geogebra{m4zhhshd}{900}{600}
\end{center}
\end{onlineOnly}





\pdfOnly{
Access Geogebra interactives through the online version of this text at
 
\href{https://www.geogebra.org/classic/t9rxbmmt}.
}
 
\begin{onlineOnly}
    \begin{center}
\geogebra{t9rxbmmt}{900}{600}
\end{center}
\end{onlineOnly}


\pskip
 
For a Desmos version of a similar animation:

\href{https://www.desmos.com/3d/7fb88aea41}{Mutually Tangent Spheres}






\end{document}

